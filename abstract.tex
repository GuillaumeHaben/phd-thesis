\chapter*{Abstract}

Many companies rely on software testing to verify that their software products meet their requirements. However, test quality and, in particular, the quality of tests interacting with the \gls{gui}, i.e. \gls{suit}, is hard to achieve. The problem becomes challenging when the \gls{sut} evolves, as \gls{suit} suites need to adapt and conform to the evolved software. Indeed, because the user interface is a part of the system that tends to undergo rapid evolutions, \gls{suit}s are particularly prompt to break.

In this dissertation, we present an empirical evaluation of the impact of the evolution and the maintenance of SUIT suites. We aim to demonstrate the problem of test maintenance and overall improve the understanding of SUIT scripts evolution. To that end, we identify, collect and analyze test code changes across the evolution of an industrial test suite. We show that the problem of test maintenance is largely due to test fragility (most commonly performed changes are due to locator and synchronization issues) and bad practices (over 30\% of keywords are duplicated). 

To further investigate the question of bad test code practices such as test clones, we perform a multivocal study to identify which bad practices, i.e. SUIT smells, are already studied in both industry and academia establish a list of 35 test code smells. For 16 of them, we derive metrics to analyze their diffusion across tests as well as potential refactoring actions removing the test code smells. Conducting a second empirical study, including both industrial and open-source test suites, we show that test code smells are largely present in \gls{suit}s, potentially contributing to the fragility and hindering the maintenance process. Interestingly, when observing refactoring actions, they remain rare during the lifespan of the tests. Yet, symptoms tend to disappear as old tests are discarded and new ones are introduced.

However, during the analysis of SUIT smells, we observe that bad practices impacting locators do not appear often in the test code. This observation contrasts with the analysis of the evolution of the SUIT presented in our first empirical study.  This apparent contradiction arises from the limitation of property-based locators which rely on the structure of the representation layer of the \gls{sut}. Thus, such approaches are sensitive to internal iso-functional changes occurring during the normal evolution of the \gls{sut}. To account for this limitation, we introduce a new way of expressing locators, HPath. Instead of relying on the internal representation of the GUI, HPath relies on its rendered characteristics. Our results suggest that despite what is presented in the literature on smells, locators relying on a smaller number of GUI elements to fully qualify a target do not necessarily lead to more robust locators. On the contrary, the choice of the GUI element properties seems to play a stronger role in the robustness to change.

Overall, this dissertation provides insights into how \gls{suit}s evolve and shows that \gls{suit} fragility plays a major role in the associated maintenance effort. It also proposes techniques that effectively facilitate the maintenance with the early detection of sub-optimal patterns and the introduction of a locator representation more robust to \gls{sut} evolution.
