\chapter{The Importance of Discerning Flaky from Fault-triggering Test Failures}
\label{chap:chromium}

\setcounter{minitocdepth}{1}
\justifying
\textit{While promising, the actual utility of the methods predicting flaky tests remains unclear since they have not been evaluated within a continuous integration (CI) process. In particular, it remains unclear what is the impact of missed faults, \ie the consideration of fault-triggering test failures as flaky, at different CI cycles. In this chapter, we apply state-of-the-art flakiness prediction methods at the Chromium CI and check their performance. Perhaps surprisingly, we find that the application of such methods leads to numerous faults missed, which is approximately \nicefrac{3}{4} of all regression faults. To explain this result, we analyse the fault-triggering failures and find that flaky tests have a strong fault-revealing capability. Overall, our findings suggest that future research should focus on predicting flaky test failures instead of flaky tests (to reduce missed faults) and reveal the need for adopting more thorough experimental methodologies when evaluating flakiness prediction methods (to better reflect the actual practice).\\
}

\chapterPage{This chapter is based on the work in the following paper:\\
\begin{itemize} 
\vspace{-2mm}
\item \fullcite{haben2023importance}
\end{itemize}}

\section{Introduction}
\label{sec:chromium-introduction}


% Continuous Integration pipeline
Continuous Integration (CI) is a software engineering process that allows developers to frequently merge their changes in a shared repository~\cite{CI}. To ensure a fast and efficient collaboration, the CI automates parts, if not all, of the development life cycle. Regression testing is an important aspect of CI as it ensures that new changes do not break existing functionality. Test suites are executed for every commit and test results signal whether changes should be integrated into the operational codebase or not.

% Flaky tests
Tests are an essential part of the CI as they prevent faults from interring the codebase, and they ensure smooth code integration and overall good software function. Unfortunately, some tests, named flaky tests, exhibit a non-deterministic behaviour as they both pass and fail for the same version of the codebase. When flaky tests fail, they send false alerts to developers about the state of their applications and the integration of their changes. 

Indeed, developers spend time and effort investigating flaky failures, as they can be difficult to reproduce, only to discover that they are false alerts ~\cite{Eck2019}. These false alerts occur frequently in open-source and industrial projects~\cite{Bell2018,Kowalczyk2020,Lam2019iDFlakies,LeongSPTM19} and make developers lose not only time but also their trust in the test signal. This trust issue in turn introduces the risk of ignoring fault-triggering test failures. This way, false alerts defy the purpose of software testing and hinder the flow of the CI.

% Existing approaches to detect flaky tests 
To deal with test flakiness, many techniques aiming at detecting flaky tests have been introduced. A basic approach is to rerun tests multiple times and observe their outcomes. While to some extent effective, test reruns are extremely expensive \cite{LeongSPTM19,Bell2018} and unsafe. To this end, researchers have proposed several approaches relying on static (the test code) \cite{camara2021use,Pinto2020,fatima2021flakify,King2018, LeongSPTM19} or dynamic (test executions) \cite{Bell2018,Lam2019iDFlakies,ziftci2020flake} information (or both) \cite{alshammari2021flakeflagger} to predict whether a given test is flaky. 

Among the many flakiness prediction methods, the vocabulary-based ones~\cite{Haben2021,Pinto2020,Camara2021VocabExtendedReplication,Bertolino2020,olewickiBrown} are the most popular \cite{ParryKHM22}. They rely on machine learning models that predict test flakiness based on the occurrences of source code tokens of the candidate tests. Interestingly, previous research has found these approaches particularly precise, with current state-of-the-art achieving accuracy values higher than 95\% \cite{Pinto2020,fatima2021flakify,camara2021use,Camara2021VocabExtendedReplication,Haben2021}. 

At the same time, vocabulary-based approaches are static and text-based, thus, they are both portable, \ie limited to a specific language, and interpretable, \ie users may understand the cause of flakiness based on the keywords that impact the model's decisions. All these characteristics (precision, portability and interpretability) make vocabulary-based approaches appealing; they are flexible and easy to use in practice. In view of this, we decided to replicate these techniques on an industrial project (the Chromium project) and evaluated their ability to effectively support the detection of flaky tests during the CI operation cycles. %, i.e., flag flaky test failures instead of fault-triggering ones.

Perhaps not surprisingly, we found a similar prediction performance (99.2\% precision 98.4\% recall) to the ones reported by previous studies. Surprisingly though, we noticed numerous fault-triggering failures (approximately 76\% of all fault-triggering failures) being marked as flaky by these prediction methods. This means that, in the case where the Chromium teams were to follow the recommendations of these techniques, they would have missed at least 76\% of all regression faults (considering their respective test failures as flaky) that were captured by their test suites. 

The %paradox of having high flaky test prediction precision and 
significantly  high fault loss experienced (fault-triggering failures considered as flaky) motivated the investigation of the fault-triggering failures. To this end, we made the following three findings: 
\begin{itemize}
    \item \textit{Flaky tests have a strong fault-revealing capability}, \ie more than \nicefrac{1}{3} of all regression faults are triggered by a test that exhibits flaky behaviour at some point in time. This means that methods aiming at detecting flaky tests, inevitably flag as flaky fault-triggering test failures made by these tests. This indicates an inherent limitation of all methods focusing on identifying flaky tests, a fact that is largely ignored by previous studies.
    
    \item \textit{Many fault-revealing tests have characteristics similar to flaky ones} and thus, are mistakenly flagged as flaky by the flakiness prediction methods. %While such mistakes are expected, given the predictive nature of the approaches, these are expected to be low given the 99.2\% prediction precision we have. However, 
    In our data, the set of fault-triggering failures made by non-flaky tests that is mistakenly predicted as flaky represents 56.2\% of all fault-triggering failures. 

    \item \textit{The majority of the flaky tests fail frequently (87.9\% of the flaky tests have also flaked in the past)}, making prediction methods mark them as flaky based on the test history, instead of the characteristics of their failures. In particular, a dummy method that classifies as flaky any test that flaked at least once in the past achieves precision and recall values of 99.8\% and 87.8\% when predicting flaky test failures.%, explaining the high accuracy we find. %However, in practice developers need to minimize both the number of fault-triggering failures missed and false alarms. 
\end{itemize}

The above findings motivate the need for techniques focusing on flaky test failures, instead of flaky tests, \ie discriminating between fault-triggering and flaky test failures, an essential problem that has largely been ignored by previous research \cite{Parry2021}. Therefore, we adapt the vocabulary-based methods for failure-focused predictions and check their performance. Although they miss fewer fault-triggering test failures than the test-focused methods (20.3\% FPR compared to 76.2\% FPR), their ability to detect fault-triggering failures remains poor, MCC value of 0.25. 

With the hope of improving the methods' performance, we augment their feature set with dynamic features related to (flaky) test executions (\eg run duration and historical flake rate) that we find useful for flakiness diagnosis. We achieve a better --but still not acceptable performance -- MCC value of 0.42, indicating an improvement to detect fault-triggering failures when considering dynamic features. 

Overall, our study demonstrates the need for methods that can effectively predict flaky test failures (instead of flaky tests), to reduce the faults missed, and the need for adopting more thorough experimental methodologies, reflecting the needs of the domain of the actual practice (not just classification metrics), when evaluating flakiness predictions. 

In summary, the contributions of our paper are:
\begin{itemize}
    \item We present \textit{a large empirical study on flakiness prediction, based on the Chromium project} -- one of the biggest open-source industrial projects -- involving 10,000 builds, more than 200,000 unique tests and 1.8 million test failures. Our study is the first to study the suitability of applying flakiness prediction into a CI pipeline by focusing on the potential losses that they introduce: missed fault-triggering failures.  
    
    \item We provide empirical evidence that flaky test prediction methods, despite being very precise, are practically non-actionable since they \textit{flag as flaky a majority, approximately 76.2\%, of all fault-triggering failures} (56.2\% due to misclassifications of fault-revealing tests and 20\% due to correct classification of flaky tests that reveal faults).

    \item We provide empirical evidence that \textit{flaky tests have strong fault-revealing capabilities}, indicating an inherent limitation of existing methods. At the same time, our results motivate the need for failure-focused prediction techniques. Unfortunately, we also show that existing vocabulary-based methods are insufficiently precise, calling for additional research in this area.  

    \item We investigate whether \textit{imbuing the training data with additional dynamic features can enhance the failure prediction effectiveness}. These results show  improvement (MCC values are up to 0.42), but indicate that more work remains to be done to develop deployable flakiness prediction for real-world systems.
   
\end{itemize}





% They can be divided into three families: static, dynamic and hybrid. Static approaches do not require test executions. They often have the goal of remaining lightweight, \ie using features that are easily retrievable such as code vocabulary~\cite{Haben2021,Pinto2020,Camara2021VocabExtendedReplication} and test smells~\cite{camara2021use,Pontillo}. 
% %or others~\cite{fatima2021flakify,King2018}. 
% Dynamic approaches require test executions, which tend to yield better detection results, but also incur more costs and usability challenges. 
% % Notably, Deflaker~\cite{Bell2018} and IDFlakies~\cite{Lam2019iDFlakies} are designed to use a minimal number of reruns. ShakeIt~\cite{Silva2020} identifies flaky tests by stressing the execution environment. Studies making use of code coverage information also emerged~\cite{FlakinessGoogle,Shi2019Mitigating}. 
% Finally, hybrid approaches are relying both on static features and dynamic features like FlakeFlagger~\cite{alshammari2021flakeflagger}.

% Problematic
%Because flaky tests are hard to fix~\cite{Eck2019,Habchi2022Qualitative}, developers typically quarantine them or ignore their outcome when running the whole test suite on a modified code base. This practice is risky because important information about the quality and correctness of the code covered by flaky tests can be lost. In particular, regressions (faults) can be missed, leading to the deployment of a buggy version of the software in production. Perhaps worse, the prediction approaches based on machine learning are inherently imperfect (due to their statistical nature) and may accentuate the number of missed regressions by flagging benign tests as flaky.


% Goal of the paper
%In this paper, we seek to investigate, in a real-world case, to what extent the prediction and quarantine of flaky tests affect the effectiveness of regression testing. We focus our case study on the Chromium project -- one of the biggest open-source projects with open access to the CI history including test results. We mined 10,000 Chromium builds from the CI history, totalling more than 200,000 unique tests and 1.8 million test failures.

%We first investigate the prevalence of critical cases where regressions are identified by flaky tests and the number of concerned builds. Special consideration should be addressed in those perilous cases as flaky test signals are known to be neglected. Next, as research mostly focuses on the binary classification of flaky tests and non-flaky tests, we evaluate the performance of the vocabulary-based methods to detect flaky tests in Chromium. More specifically, we are interested in the number of false positives -- benign tests predicted as flaky -- because wrongly quarantining them would increase the critical signals ignored by developers.

%The paradox between the false alarms raised by flaky tests and the risk induced by ignoring them leads us to consider the prediction of \emph{flaky failures} as an essential problem that past research has largely ignored \cite{Parry2021}. We, thus, evaluate whether the same approaches for flaky test prediction can be tuned (by adding features specific to each test execution in addition to traditional features) to predict flaky failures. This would reveal an apparent proximity (or lack thereof) between the two problems and assess the need for dedicated solutions.

%Overall, our study investigates and answers the following research questions:
%\begin{itemize}
%\item \textbf{RQ1:} How often are regressions identified by flaky tests? \\
%\textbf{Findings:} We report a high number of flaky tests in all builds (178 on average). 1,766 builds include regressions that are detected only by flaky tests and this represents 73\% of all builds with regressions. Meanwhile, out of the 2,343 tests revealing regressions, 897 (i.e. 38\%) are flaky. This demonstrates that ignoring flaky tests entirely would result in missing a significant number of regressions. These results combined also reveal that when a flaky test reveals a specific regression, it is the only test that does so.

%\item \textbf{RQ2:} How accurate are flaky test prediction methods on Chromium? \\
%\textbf{Findings:} The vocabulary-based model we built (replicating previous work) has a precision of 22\%, a recall of 88\% and an MCC of 34\%. The model is able to find a large proportion of flaky tests but mislabels 32\% of all non-flaky tests as flaky. This is disastrous because developers blindly applying prediction methods to ignore flaky tests could miss many cases of regression. More generally, the performance of prediction methods is below the ones reported in previous studies~\cite{Pinto2020,Haben2021,Camara2021VocabExtendedReplication}, indicating that Chromium is a more challenging ground for flaky test prediction.

%\item \textbf{RQ3:} How accurately can we predict flaky failures from fault-triggering failures? \\
%\textbf{Findings:} We show that solely relying on static features (previously used for flaky test prediction) poorly predicts flaky failures. Our model obtained an MCC of 4\%. Adding dynamic features related to (flaky) test executions (\eg run duration and flake rate) improves the performance with an MCC of 20\% but remains unacceptable. This demonstrates that the same solutions used to identify flaky tests are not applicable to this new problem, and raises the need for new research approaches for flaky failure prediction.
%\end{itemize}

%%% INSERT here a concluding paragraph like "Overall, through our study we... [demonstrate/shed light on] ... + key take away message for research. => Perhaps @MIKE will have a nice formulation?

To support future research, we share our dataset, experimental data and related code, in a replication package.\footnote{https://anonymous.4open.science/r/ChromiumFlakyFailures}

%The remaining of this paper is organised as follows: 
%Section~\ref{section:related} discusses related works.
%Section~\ref{section:chromium} presents the Chromium CI and Section~\ref{sec:dataset} our data collection process.
%Section~\ref{sec:evaluation} presents the different research questions in details and the methodology we used to answer them.
%Section~\ref{sec:results} presents the case study results. 
%Section~\ref{sec:discussion} discuss the implication of the results in more depth. Section~\ref{sec:threats} discusses the threats to validity and finally, Section~\ref{sec:conclusion} concludes with the main findings.
\section{Chromium}
\label{sec:chromium-chromium}


\subsection{Overview}

% General Information about Chromium
Started in 2008, with more than 2,000 contributors and 25 million lines of code, the Chromium web browser is one of the biggest open-source projects currently existing. Google is one of the main maintainers, but other companies and contributors are also taking part in its development.

% LuCI hierarchy
Chromium relies on \textit{LuCI} as a CI platform~\cite{onlineChromiumGithub}.
It uses more than 900 parallelized builders, each one of them used to build Chromium with different settings (\eg different compilers, instrumented versions for memory error detection, fuzzing, etc) and to target different operating systems (\eg Android, Mac OS, Linux, and Windows). 

Each builder is responsible for a list of builds triggered by commits made to the project. If a builder is already busy, a scheduler creates a queue of commits waiting to be processed. This means that more than one change can be included in a single build execution if the development pace is faster than what the builders can process. Within a build, we find details about build properties, start and end times, status (\ie pending, success or failure), a listing of the steps and links to the logs. 

% Testing infrastructure
At the beginning of the project, building and testing were sequential. Builders used to compile the project and zip the results to builders responsible for tests. Testing was taking a lot of time, slowing developers' productivity and testing Chromium on several platforms was not conceivable. A swarming infrastructure was then introduced in order to scale according to the Chromium development team's productivity, to keep getting the test results as fast as possible and independently from the number of tests to run or the number of platforms to test. Currently, a fleet of 14,000  build bots runs tasks in parallel. This setup helps to run tests with low latency and handle hundreds of commits per day~\cite{TheChromiumProjects}.

% Figure Decision tree
\begin{figure}[ht]
\centering
\includegraphics[width=0.8\textwidth]{figures/chromium/gui_process_final.png}
\caption{Decision tree representing how test outcomes are determined in a build by the Chromium CI. \includegraphics[scale=0.2]{figures/chromium/pass.png} PASS depicts successful tests, \includegraphics[scale=0.2]{figures/chromium/flaky.png} FLAKY depicts tests that passed after failing at least once, while \includegraphics[scale=0.2]{figures/chromium/fail.png} UNEXPECTED depicts tests that persistently failed.}
\label{fig:decision-tree}
\end{figure}

% Information about tests
In this study, we focus on testers, \ie builders only responsible for running tests. They do not compile the project: when triggered, they simply extract the build from their corresponding builder and run tests on this version. At the time of writing, we found 47 testers running Chromium test suites on distinct operating systems versions. About 200,000 tests are divided into different test suites, the biggest ones being \textit{blink\_web\_tests} (testing the rendering engine) and \textit{base\_unittests} with more than 60,000 tests each.

% Build results
For each build performed by any tester, we have access to information about test results. Figure~\ref{fig:decision-tree} illustrates the decision process followed by LuCI to determine a test outcome in a specific build. A test is labelled as \textit{pass} when it successfully passed after one execution. In case of a failure, LuCI automatically reruns the test up to 5 times. If all reruns fail, the test is labelled as \textit{unexpected} and will trigger a build failure. In the remaining, we will be referring to \textit{unexpected} tests as \textit{fault-revealing tests}. If a test passes after having one or more failed executions during the same build, it is labelled as \textit{flaky} and will not prevent the build from passing. 


\subsection{Example of a Flaky Test}

% Figure Decision tree
\begin{figure}[!ht]
\centering
\includegraphics[width=0.6\textwidth]{figures/chromium/flakyTestExample.png}
\caption{An example of a flaky test caused by a timeout. The test consists of an HTML file \texttt{printing/webgl-oversized-printing.html}, build 119,039 of the Linux Tester. The call to \textsc{waitUntildone()} on line 19 is likely the reason for the failure.}
\label{fig:example}
\end{figure}

Figure~\ref{fig:example} shows a flaky test found in build 119,039\footnote{https://ci.chromium.org/ui/p/chromium/builders/ci/Linux\%20Tests/119039/} of the Linux Tester. This test, \texttt{printing/webgl-oversized-printing.html}, ensures that no crash happens on the main thread of the rendering process when using the system. Unfortunately, on its first execution, the test failed after 31 seconds. The run status indicates that a \textsc{TIMEOUT} happened. On the second execution, the test passed after 15 seconds and thus was labelled as flaky. In this case, an issue has been opened in Chromium's bug tracking system.\footnote{https://bugs.chromium.org/p/chromium/issues/detail?id=1393294} Developers state that \textit{"this test makes a huge memory allocation in the GPU process which intermittently causes OOM and a GPU process crash"}.

\textsc{Timeout} is a run status that intuitively leads to possible flakiness, as we can easily think of other executions of the same test being completed before reaching the time limit. In addition to this feature, we can also look for hints of flakiness in the source code. As with many UI tests in Chromium, this one is handled by \texttt{testRunner}: a test harness in charge of their automatic executions. We can see the \texttt{testRunner} making a call to \texttt{waitUntilDone()} on line 19. Vocabulary about waits is common in Chromium's web tests. This keyword, for example, could potentially be leveraged by flakiness detectors to classify tests or failures.



\section{Data}
\label{sec:chromium-data}

\subsection{Definitions}

%In this paper, we discuss many different entities at different levels: builds, tests, failures... 
Some of our definitions are slightly different from the ones used by previous work since Chromium has its specific continuous integration setup. To make things clear, we define the elements we will discuss in this section.  
In the scope of \emph{a single build}, we employ the following definitions: 
\begin{itemize}
  %  \item \textbf{Regression}: A software fault caused by a change in the code.
    \item \textbf{Fault-revealing test}: A test that consistently failed after reruns in the same build, revealing a regression fault.
    \item \textbf{Flaky test}: A test that failed once or more and then passed after reruns in the same build. 
    % \item\textbf{Fault-revealing flaky test}: A fault-revealing test that was found to be flaky in at least one other build.\footnote{Note that in a given build, a fault-revealing flaky test always fail and cannot be labelled as a flaky test within the same build.} 
    \item\textbf{Flaky failure}: A test failure caused by a flaky test.
    \item\textbf{Fault-triggering failure}: A test failure caused by a fault-revealing test.\\
\end{itemize}


\subsection{Data collection}

To perform our study, we collected test execution data from 10,000 consecutive builds completed by the Linux Tester by querying the LuCI API. This represents a period of time of about 9 months taken between March 2022 and December 2022. 

Table ~\ref{table:infoRuns} summarises the list of information extracted and computed for all tests executed in all builds. The \textsc{buildId} corresponds to the build in which tests were executed. \textsc{runDuration} is the execution time spent to run the test. \textsc{runStatus} gives information about the run result (\eg passing, failing, and skipped) and \textsc{runTagStatus} returns more precise information about the result of a run depending on the type of test or test suite (\eg timeout and failure on exit). We retrieved information about the tests' source code by querying Google Git~\footnote{https://chromium.googlesource.com/chromium/src/+/HEAD/}. As builds often handle several commits, we select the revision corresponding to the head of the blame list: the one on which tests were executed. The \textsc{testSuite} is simply the name of the test suite the test belongs to and \textsc{testId} is a unique identifier for a test composed of the test suite and the test name (the same test name can be present in different test suites).

All the scripts used to collect the data alongside the created dataset are available in our replication package.


% Table: Description of the features
% Taken from the README
\begin{table}[ht]
%\vspace{1.0em}
\centering
\caption{Description of our features. Column \textit{Feature Name} specifies the identifiers used in our dataset, while Column \textit{Feature Description} details the features}
%\vspace{-1em}
\label{table:infoRuns}
\begin{tabular}{ l | l } 
\toprule
\textbf{Feature Name} & \textbf{Feature Description} \\
\midrule
buildId & The build number associated \\
{} & with the test execution \\
\midrule
flakeRate & The flake rate of the test over the last\\  
{} & 35 builds \\
\midrule
runDuration & The time spent for this test execution \\
\midrule
runStatus & ABORT\\ {} & FAIL\\ {} & PASS\\ {} & CRASH\\ {} & SKIP \\
\midrule
\multirow[t]{9}{*}{runTagStatus} & CRASH\\ {} & PASS\\ {} & FAIL\\ {} & TIMEOUT\\ {} & SUCCESS \\
{} & FAILURE\\ {} & FAILURE\_ON\_EXIT\\ {} & NOTRUN\\ {} & SKIP\\ {} & UNKNOWN \\
\midrule
testSource & The test source code \\
\midrule
testSuite & The test suite the test belongs to \\
\midrule
testId & The test name \\
\bottomrule
\end{tabular}
\vspace{1.0em}
\end{table}

% Table: Data collected from the Chromium CI
% Script: infoDataset.py
\begin{table*}[ht]
%\vspace{1.0em}
\centering
\caption{Data collected from the Chromium CI. We used the \textit{Linux Tests} tester, with \textit{10,000} Builds mined over \textit{nine} months. We extracted Passing, Flaky and Fault-revealing tests and their associated Flaky and Fault-triggering Failures.}
\vspace{-0.5em}
\label{table:infoDataset}
\begin{tabular}{ l | c | c c | c c c | c c } 
\toprule
\multirow{2}{*}{\textbf{Tester}} & \multirow{2}{*}{\textbf{Nb of Builds}} & \multicolumn{2}{c|}{\textbf{Period of Time}} & \multicolumn{3}{c|}{\textbf{Number of Tests}} & \multicolumn{2}{c}{\textbf{Number of Failures}} \\
{} & {} & \textbf{From} & \textbf{To} & \textbf{Passing} & \textbf{Flaky} & \textbf{Fault-revealing} & \textbf{Flaky} & \textbf{Fault-triggering} \\
\midrule 
Linux Tests & 10,000 & Mar 2, 2022 & Dec 1, 2022 & 198,273 & 23,374 & 2,343  & 1,833,831 & 17,171 \\
\bottomrule
\end{tabular}
\end{table*}

\subsection{Computing the flake rate}
\label{sec:testHistory}

The historical sequence of test results is a valuable piece of information commonly used in software testing at scale~\cite{LeongSPTM19,Kowalczyk2020}. We analyse the history of fault-revealing tests and flaky tests by relying on their \textsf{flakeRate}.

This means that for a test $t$ failing (due to flaky or fault-triggering failure) in a build $b_{n}$, we analyse all the builds from a time window $w$ (\ie from $b_{n-w}$ to $b_{n-1})$ to calculate its flake rate as follows: \\

\noindent\begin{minipage}{\linewidth}
\begin{equation}
  flakeRate(t,n) = \frac{ \sum_{x=n-w}^{n-1} flake(t,x) } {w}
\end{equation}
\end{minipage}%
\\

where $flake(t,x) = 1$ if the test $t$ flaked in the build $b_{x}$ and 0 otherwise.
This metric allows us to understand if the flakiness history of a test can help in the flakiness prediction tasks.
The test execution history (\aka heartbeat) has been used in multiple studies (especially industrial ones~\cite{Kowalczyk2020,LeongSPTM19}) to detect flaky tests.
These studies assume that many flaky tests have distinguishable failure patterns over builds and hence can be detected by observing their history.
We check whether this assumption also holds in the case of Chromium.

\begin{figure}[!htbp]
    \vspace{-1em}
  \centering
    \includegraphics[width=0.8\textwidth]{figures/chromium/densityFlakeRate.png}
    \vspace{-1em}
    \caption{Flake rate (\textit{x-axis}) for \textcolor{orange}{Flaky} and \textcolor{red}{Fault-revealing tests}. Density (\textit{y-axis}) is the probability density function. The area under curves integrates to one. Many flaky tests are always flaky in their previous builds. A majority of fault-revealing tests have no history of flakiness at all.}
    \label{fig:kdeRates}
\end{figure}

To illustrate the flake rate differences between flaky and fault-revealing tests, we plot the flake rate for both test categories in Figure~\ref{fig:kdeRates}. The flake rate is computed using a window of 35 builds. To set this time window, we checked the number of flaky tests having a $flakeRate() = 0$ for build windows ranging from 0 to 40 builds with a step of 5. We observed a convergence at size 35, meaning that higher numbers of builds do not provide additional information.

In the majority of cases, flaky tests have a history of flakiness: the percentage of flaky tests having a $flakeRate() > 0$ is in fact $87.9\%$. Furthermore, we see a pike for $flakeRate() == 1$, 9.5\% of flaky tests were always flaky in their 35 previous builds. 
Still, there is a non-negligible amount (45.3\%) of fault-revealing tests that were flaky at least once in previous builds considered: with a $flakeRate() > 0$.
From these observations, we may suggest that the \textsf{flakeRate()} can be used to detect flakiness.
Nevertheless, there is still an important overlap between the history of flaky tests and fault-revealing tests. 


\section{Objectives and Methodologies}
\label{sec:chromium-objectives}


\subsection{Research questions}

We start our analysis by assessing the effectiveness of the existing flakiness prediction methods in our project by considering the critical cases where fault-revealing test failures are flagged as flaky by the prediction methods in various CI cycles. We thus are interested in investigating the methods' performance under realistic settings, i.e., correctly detected and missed flaky and fault-triggering test failures, when trained with past CI data and evaluated on future ones. In contrast to previous work, this analysis introduces a new dimension in the evaluation of flakiness predictions which is the investigation of what we lose when adopting a prediction method (the fault-triggering failures classified as flaky). Therefore, we ask:

\begin{description}
\item[RQ1:] How well do flaky test prediction methods discern flaky test failures from fault-triggering ones? 
\end{description}

To establish realistic settings, we train the prediction models using the information available (flaky tests and non-flaky ones) at a given point in time, where we have sufficient historical data to train on. We then evaluate the models in subsequent builds with respect to flaky and fault-triggering test failures. 
% To avoid coincidental results, we repeat the process multiple times at different points in time, i.e., with different training and evaluation data. 
We replicate the vocabulary-based methods since they are popular, easy to implement and quite effective, and aimed at learning to predict flaky tests, as done by previous studies.

After checking the prediction performance in a realistic setting (test failures), we repeat the entire process but now we train on historical test failures instead of tests. We make this adaptation with the hope of improving further our predictions and perhaps improving our understanding of the impact that such predictions may have on missed fault-triggering test failures (those marked by the models as flaky). Hence, we ask: 

\begin{description}
\item[RQ2:] How well do flaky test failure prediction methods discern flaky test failures from fault-triggering ones? 
\end{description}

Finally, we wish to be comprehensive, so we also optimize and extend the prediction methods with additional features, some of which were suggested by previous studies (the flake rate~\cite{Kowalczyk2020}, the run duration~\cite{alshammari2021flakeflagger}) and some dynamic features (test run status, test run tag status, test run duration) that we found by us when experimenting with the flaky tests. Thus, we ask:

\begin{description}
\item[RQ3:] Can we improve the accuracy of the flaky test failure predictions by considering dynamic test execution features?
\end{description}

To answer RQ3, we repeat the analysis carried out for RQ2 but now we are training and optimising for the additional features that we determined during our analysis and check the performance we achieve with regards to test failures, as performed in RQ2. 



%With this case study, we first seek to understand the prevalence of flaky tests in the Chromium project. We are especially interested in the critical cases where regressions are reported by flaky tests. Then, we aim at contrasting two different tasks:

%\begin{itemize}
 %   \item \textbf{Objective I: Discerning flaky tests from non-flaky tests.} Broad attention is currently given by the research community towards the problem of detecting flaky tests. The motivation is the following: Flaky tests represent a major concern in software testing as they force developers to investigate false alerts and rerunning tests can be costly. Training a classifier to predict flaky tests would help focus on tests that are the most likely to be flaky first, or could be used as an alternative to reruns.
%    \item \textbf{Objective II: Discerning legitimate failures from flaky failures.}
%    Fewer studies are addressing the problem of classifying legitimate failures and flaky failures. However, our findings lead us to focus on failure detection. It is the execution of a test and its context that needs to be considered as the same test could lead to a flaky failure in one build and to a legitimate failure in another. An efficient solution to this problem has the potential of reducing rerunning costs and helping developers get better insights into their test outcomes.
%\end{itemize}


\subsection{Experimental procedure}

\subsubsection{Selection of a flaky test detection approach} 
\label{sec:evaluationSelection}
Being a recent topic of interest, several techniques have been introduced in the scientific literature. Approaches relying on code coverage such as FlakeFlagger~\cite{alshammari2021flakeflagger} or DeFlaker~\cite{Bell2018} are challenging to implement in our case. Chromium's code base consists of several languages and code coverage is both costly and non-trivial to retrieve. Test smells~\cite{camara2021use} approaches are also difficult to extract as tests are written in many different languages and tools do not always exist. 
% Another recent fully-static approach that was presented in the literature is Peeler~\cite{qin2022peeler}. Peeler's features rely on test dependency graphs that are generated using contextual paths between each test and their code under test. It was implemented in Java. Unfortunately for us, Java is not a prevalent language in Chromium's tests. 
Having those constraints in mind, we decide to use the vocabulary-based approach introduced by Pinto~\etal ~\cite{Pinto2020}. It received significant attention with several replication studies conducted~\cite{Haben2021,Camara2021VocabExtendedReplication} and follow-up studies and its portability makes it easy to implement regardless of the languages being used.

\subsubsection{Training and validation of the existing approaches (RQ1)} 
We evaluate the ability of the vocabulary-based approach, trained to differentiate flaky from non-flaky tests and used to predict flaky test failures. To do so, we divide our dataset into a training set (containing test information about the first 8,000 builds) and a test set (containing test information about the last 2,000 builds). We train our model following the existing methodologies. The flaky set includes all tests marked as flaky in the training set. 
The non-flaky set includes all fault-revealing tests and all passing tests in the 8,000\textsuperscript{th} build (\ie the last build of the training set) minus the tests that are found as flaky in any of the builds under study (to increase the confidence of being non-flaky). The test set includes all flaky test failures and all fault-triggering failures (reported by fault-revealing tests). The test set is common in all RQs.
% We discuss performances in two scenarios: the first, where we keep duplicated tests and the second, where we remove them. Usually, when evaluating a machine learning model, we remove duplicated data points to evaluate its ability to work on unseen data. In our case, a large part of the dataset consists of duplicated tests, as the same tests are executed in every build (with few additions and deletions brought by development). We believe the two scenarios are useful to report, as in a practical setup, the model would be given tests already seen in the past in most cases, but we still want to understand if feature trends can be found and used to identify unseen tests.

\subsubsection{Implementation of a failure classifier (RQ2 and RQ3)}
We select flaky failures in our dataset as all failures produced by flaky tests and fault-triggering failures are all failures produced by fault-revealing tests. There are no duplicated data in the case of test failures, as each test execution is unique. For RQ2, we train our classifier on non-flaky executions (passing and fault-revealing tests execution) and flaky failures. In RQ3, we report the performance of a model using execution features (run duration and flake rate).

\subsubsection{Time-sensitive evaluation}
We split our data in two parts: the first 80\% builds are selected as a training set and the last 20\% as a holdout set. By doing so, we respect the evolution of failures across time and avoid any data leakage that could occur by randomly selecting data. This time-sensitive aspect is very important to consider. We found that not taking this condition into account and training a model on a shuffled dataset would greatly overestimate the performance. Figure~\ref{fig:dataset} shows a representation of our dataset. Flaky tests are present in all builds and Fault-revealing tests are occasional: they happen in \nicefrac{1}{4} of builds (See Section~\ref{sec:chromium-discussion}). To mitigate imbalance, we collected all passing tests for 1 build: $b_{8,000}$ and use them in our set of non-flaky tests, for training.

% Figure dataset
\begin{figure*}[t]
\vspace{-1.4em}
\centering
\includegraphics[width=0.8\textwidth]{figures/chromium/dataset.png}
\vspace{-1.1em}
\caption{The data collected from Chromium's CI consists of flaky, fault-revealing and passing tests spread across 10,000 builds. The build timeline ranges from build \textit{$b_0$} to \textit{$b_{10,000}$} and depicts the distribution of the collected tests: flaky tests are spread across all builds and fault-revealing tests happen occasionally. Due to a large number of passing tests, we collected them from the \textit{$b_{8,000}$} build (\ie at the end of our training set).}
\label{fig:dataset}
\vspace{-0.2em}
\end{figure*}

\subsubsection{Classifier selection and pipeline description}
% Model 
We use a random forest classifier to perform the predictions. Unfortunately, our dataset is imbalanced with the minority class being 1\% of the data. Using a simple random forest would greatly increase the chance of having few or no elements from our minority class in the different trees, making the overall model poor in predicting the class of interest. To alleviate this issue, we decide to use a Balanced Random Forest classifier\cite{chen2004using} to facilitate the learning. This implementation artificially modifies the class distribution in each tree so that they are equally represented. Furthermore, we use SMOTE in the training phase to augment data for the minority class~\cite{smote}.

% Test source representation
To represent the tests, we use \textit{CountVectorizer} to convert texts as a matrix of token counts. This technique, known as bag-of-words, is used in previous vocabulary-based approaches~\cite{Pinto2020,Haben2021,Camara2021VocabExtendedReplication,Bertolino2020,olewickiBrown}. These vectors initially contain as many features as the words appearing in source code of the tests. As the generated dictionary can become big (in terms of size) we need to use feature selection to reduce it, remove irrelevant features (reducing noise in the data) and select the most informative features. Feature selection is thus, helping to reduce the model training time and to improve the overall performance and interpretability of the model. 

We use SelectKBest\cite{selectkbest} which retains the $k$ highest score features based on the univariate statistical test $\chi^2$.
Hyper-parameters of the machine learning pipeline, \ie the number of trees in the forests, the sampling strategy for SMOTE and the number of features to be retained are tuned using a grid search approach and cross-validation in the training set. Once optimized, we retrain a model fitted on the whole training set and evaluate it on the holdout set.

\subsubsection{Metrics}
Finally, to evaluate the different models, we rely on the following metrics derived from true positives (TP), true negatives (TN), false positives (FP) and false negatives (FN):
    \[
    \textbf{Precision} = \frac{TP}{TP+FP} \quad \quad \quad \textbf{Recall} = \frac{TP}{TP+FN}
    \]
The accuracy of a model is sensitive to class imbalance. In particular, the precision and recall metrics can easily be impacted when one class is underrepresented. To alleviate this issue, we report the Matthews Correlation Coefficient (MCC) which is a more reliable statistical rate to avoid over-optimistic results in the case of an imbalanced dataset \cite{chicco2020advantages}.
This metric takes into consideration all four entries of the confusion matrix. MCC ranges from -1 to 1 and is given by the following formula: 
    \[
    \textbf{MCC} = \frac{TN \times TP - FP \times FN}{\sqrt{(TN+FN)(TP+FP)(TN+FP)(FN+TP)}}
    \]
In addition to those metrics, we also report the false positive rate (FPR), that is, the ratio of fault-triggering failures misclassified as flaky over all fault-triggering failures. It is defined by:
    \[
    \textbf{FPR} = \frac{FP}{FP+TN}
    \]
%In our case, the FPR is the proportion of fault-revealing tests incorrectly classified as flaky (\ie missed faults). The FNR is the proportion of flaky tests incorrectly classified as fault-revealing (\ie false alerts). 
\section{Experimental Results}
\label{sec:chromium-results}


%In this section, we report the results of each research question.

\subsection{RQ1: Discerning flaky from fault triggering test failures when training on tests}

We trained a model on 69,159 passing tests, 910 fault-revealing tests and 8,857 flaky tests (unique tests).
Then, we evaluated it on 217,503 failures caused by flaky tests and 2,320 fault-triggering failures caused by fault-revealing tests. Table ~\ref{table:rq1} reports the obtained performance. Similar to the performance achieved by previous vocabulary-based models on other datasets, our model was able to reach high accuracy with a precision of 99.2\% and a recall of 98.9\%. However, we note a high false-positive rate. This is due to an important amount (76.2\%) of fault-triggering failures classified as flaky (FP). This is concerning: fault-triggering failures should not be misclassified as they reflect the existence of real faults. Overall, the MCC value is equal to 0.20, which is relatively low and shows that the model struggles (compared to random selection) to identify fault-triggering failures. 

\begin{table}[ht]
\caption{Vocabulary-based model performance for the prediction of \textit{flaky test failures vs fault-triggering failures} when trained on flaky vs non-flaky (fault-revealing and passing tests). Despite a high accuracy on flaky failures, the low MCC and high FPR show us that it remains challenging for the model to classify negative elements (in our case: fault-triggering failures)}
\vspace{-1.0em}
\label{table:rq1}
\centering
\begin{tabular}{c|c|c|c} 
 \toprule
 \textbf{Precision} & \textbf{Recall} & \textbf{MCC} & \textbf{FPR} \\ [0.5ex] 
 \midrule
 99.2\% & 98.9\% & 0.20 & 76.2\% \\ 
 \bottomrule
\end{tabular}
\end{table}


% Figure Confusion matrix
\begin{figure}[!htbp]
\centering
\vspace{-1.2em}
\includegraphics[width=0.8\textwidth]{figures/chromium/rq1.png}
\vspace{-1.3em}
\caption{Confusion matrix for the vocabulary-based model. High accuracy is reached similar to the performance reported in previous works. Nonetheless, 1,768 (76.2\%) out of the 2,320 fault-triggering failures are mislabeled as flaky.}
\label{fig:confMatrix}
\end{figure}


The confusion matrix of our model decisions is depicted in Figure ~\ref{fig:confMatrix}. The x-axis reports the predicted label and the y-axis the actual label. Correct classifications are displayed in the top left (TN) and bottom right (TP). We clearly observe that the model is able to detect flaky tests with high precision. We also see that 2,435 flaky tests are classified as non-flaky (FN). This number is also important to consider: it translates in all cases where developers will be required to investigate irrelevant failures.

We want to further understand the reasons behind the classification of fault-triggering failures. Therefore, we analyse the (fault-revealing) tests causing those failures. Out of the 2,320 fault-triggering failures, 1,768 are in the set of false positives (76.2\%) among which we found 463 (20\% of all fault-triggering failures) whose tests have a history of flakiness (flakeRate > 0) and 1,305 (56.2\% of all fault-triggering failures) without flakiness history. Here it must be noted that depending on the size of the history considered, we may have more tests with past flakiness or less. Overall in our data, 1/3 of all fault-triggering failures are due to tests that have exhibited flakiness behaviour. 


% We want to further understand the reasons behind the classification of fault-triggering failures. Therefore, we analyse the (fault-revealing) tests causing those failures. Out of the 106 unique fault-revealing tests used in our test set, there are 79 tests in the set of false positives (74\% of fault-revealing tests) among which we found 34 to be labelled as flaky tests in the training set and 12 to be labelled as fault-revealing, the 33 tests remaining are tests that do not appear in the training set (new tests). 

% This means the model misclassifies a fault-revealing test as flaky for several reasons: in 43\% of the cases it is a flaky test failing due to a fault, in 15\% of the cases, it is due to the difficulty of the model to recognise the vocabulary of flaky tests from the vocabulary of fault-revealing tests, and in the last 42\% of the cases, the test was no seen in the training data and thus, more challenging to classify. 

\begin{tcolorbox}[
    left=2pt,right=2pt,top=2pt,bottom=2pt, %margin  
    arc=0pt, % corners
    boxrule=1.2pt % line width
]
\textbf{RQ1:} Similar to previous studies, we report accurate predictions when aiming at flaky tests. However, a high proportion (76.2\%) of all fault-triggering failures is classified as flaky (missed faults) and still an important number (2,435) of flaky tests are marked as fault-triggering failures (false alerts).
\end{tcolorbox}


\subsection{RQ2: Discerning flaky from fault triggering test failures when training on test failures}

The results from RQ1 show that a vocabulary-based model trained to detect flaky tests would still yield an important number of missed faults and false alerts despite having high accuracy. Thus, our goal with RQ2 is to check whether by training our vocabulary-based model we can improve the performance of recognising fault-triggering failures. 

Table ~\ref{table:rq2and3} reports the results of such a model. In particular, the first row reports results based on failure training while the second row reports results related to RQ3. Similar to RQ1, we see a high precision and recall, 99.7\% and 91.3\% respectively, when predicting flaky failures. More interestingly, the MCC slightly increased to 0.25. 




% Table~\ref{table:rq2} shows the performance of the vocabulary-based flakiness prediction method in two scenarios. When duplicated data is left in the dataset (tests that flaked in multiple builds), the method achieves high performance with a precision of 98\%, a recall of 91\% and an MCC of 87\%. However, when we duplicated tests are removed (representing a scenario where the method is used to detect flaky tests it has not seen before), the performance drops to a precision of 22\%, a recall of 88\% and an MCC of 34\%. The detection of new flaky tests in Chromium thus remains challenging. 

% Figure Confusion matrix
% \begin{figure}[!htbp]
% \centering
% \includegraphics[width=0.5\textwidth]{img/conf_matrix.png}
% \vspace{-1em}
% \caption{Confusion matrix for the vocabulary-based model predicting flaky tests vs non-flaky tests. Importantly, 606 non-flaky tests out of the 1,863 are mislabeled as flaky tests (false positives)}
% \label{fig:confMatrix}
% \end{figure}

% Figure~\ref{fig:confMatrix} shows the confusion matrix in the no-duplicate case. We observe that flaky tests are overall correctly predicted with 168 tests properly classified as flaky and only 24 misclassified. However, 606 out of 1863 (32.70\%) of non-flaky tests are wrongly classified as flaky. This high number of false positives pose multiple threats. First, as flakiness debugging is already a tedious task for developers, giving them false positives defies the purpose of helping them regain trust in their CI. Second, and perhaps more importantly, non-flaky tests predicted as flaky might be ignored by developers in future builds where they could be fault revealing, leading to undetected regressions.

\begin{tcolorbox}[
    left=2pt,right=2pt,top=2pt,bottom=2pt, %margin  
    arc=0pt, % corners
    boxrule=1.2pt % line width
]
\textbf{RQ2:} When training on test failures, solely relying on test code vocabulary as features, to predict if a test failure is flaky or fault-triggering, model performance slightly improves but is still not effective in the context of the Chromium CI.
% Relying on flaky test predictors to identify previously-unobserved flaky tests in Chromium CI yields a 32.70\% rate of false positives. Ignoring these non-flaky tests wrongly predicted as flaky can have disastrous effects as any regression revealed by those tests would be missed.
%\textbf{RQ2:} When taking all tests into consideration, the model would perform greatly with an MCC of 87\%. However, detecting unseen flaky tests, even though feasible, is challenging as the precision remains low (22\%) despite a high recall (88\%). The high number of false positives is particularly harmful in the context of the Chromium CI.
\end{tcolorbox}

\begin{table}[ht]
\caption{Vocabulary-based model performance for the prediction of \textit{flaky failures vs fault-triggering failures} when training on flaky vs non-flaky (fault-triggering and passing test executions). The approach does not work when solely relying on static features (\ie the test source code) and is improved when considering execution features.}
%\vspace{-0.5em}
\label{table:rq2and3}
\centering
\begin{tabular}{c|c|c|c|c} 
 \toprule
 \textbf{Execution features} & \textbf{Precision} & \textbf{Recall} & \textbf{MCC} & \textbf{FPR} \\ [0.5ex] 
 \midrule
 No & 99.7\% & 91.3\% & 0.25 & 20.3\%\\ 
 Yes & 99.5\% & 98.7\% & 0.42 & 42.3\%\\ 
 \bottomrule
\end{tabular}
\vspace{-1.3em}
\end{table} 


\subsection{RQ3: Improving the accuracy of the flaky test failure predictions}

In this RQ we check the performance of the vocabulary-based models on the failure classification task when considering additional features from the test executions (run duration, and tests' historical flake rate). These features reflect better the characteristics of the test executions and are linked with test flakiness thereby leading to better results. 

In particular, the second row of Table~\ref{table:rq2and3} reports the related performance. We observe an improvement compared to the model that relies only on vocabulary (RQ2). This new model achieves a similar precision and recall of 99.5\% and 98.7\% and an improved MCC value 0.42, indicating a better performance in comparison to randomly picked selections. We see that the FPR increased to 42.3\%. Together, the results can be explained by fewer false alerts: flaky failures being marked as fault-triggering by the model. 
% This can be explained by our RQ1 results, where we show that 38.28\% of tests that are fault-revealing in at least one build are flaky in at least one other build. Because the vocabulary approach, by construction, does not distinguish different executions of the same test, it misclassifies a subset of the failures of these fault-revealing flaky tests. Adding execution-related features improves the results, though the overall performance remains low (10\% precision, 39\% recall; and 20\% MCC). 

\begin{tcolorbox}[
    left=2pt,right=2pt,top=2pt,bottom=2pt, %margin  
    arc=0pt, % corners
    boxrule=1.2pt % line width
]
\textbf{RQ3} When equipped with execution-related features, the vocabulary-based prediction methods do a better job of distinguishing flaky failures from fault-triggering failures (0.42 MCC). Still, the need remains for dedicated methods to successfully learn this challenging classification task.

%Compared to test-focused detection, it appears more challenging to distinguish flaky failures from legit failures. We can improve a bit the performance by adding dynamic features, \ie properties of the test execution. However, further research is still required to improve our ability to discern failure types.
\end{tcolorbox}
\section{Discussion}
\label{sec:chromium-discussion}


% Overall, we see that test flakiness affects Chromium's CI as in other large software systems. It is now well-established thanks to previous research studies~\cite{Luo2014,Lam2019RootCausing} and industrial reports~\cite{Micco2017,FlakinessSpotify} that flaky failures force developers to spend time on false alerts which are difficult to reproduce and debug. As they lose trust in their test suite results, they stop relying on test signals and end up integrating buggy features in the main codebase. 
% In an attempt to mitigate the problem, important attention has been given by the research community towards flaky test detection, with the goal of helping developers find flaky tests more efficiently than with the rerunning approach. \\
% \revise{\textbf{\textit{On the importance to practitioners:}}} In our case study, we make an important finding: flaky tests are not always giving false signals, as a substantial \nicefrac{1}{3} of regressions are reported by tests that happened to flake in other builds. 
% This is even more problematic when considering the fact that \nicefrac{3}{4} of failing builds only list flaky tests as fault-revealing tests. This means that signals from flaky tests should be considered and treated carefully. 
% We also noted a very low precision of our model in RQ1. This means that there are many false positives (\ie non flaky tests mislabelled as flaky tests). This is a worrying aspect in the two scenarios such a model can be used in the CI: (1) if the classifier is used to discard information coming from flaky tests, it would convince developers to not consider important signals (\eg a fault-revealing test). (2) If the model is used to find flaky tests to help developers debug them, this would require them to investigate many tests that are actually not flaky. 
% Our findings diminish the purpose of flaky test detection in the context of continuous integration: that is when detection tools are used to predict if tests are flaky or not. On the contrary, it advocates the use of tools towards the detection of failures when they occur and being able to classify them as flaky or legitimate.

% \subsection{Prevalence of regressions identified by flaky tests} 
We seek to better understand the results showing that existing approaches targeting the detection of flaky tests missed a non-negligible part (76.2\%) of fault-triggering failures by classifying them as flaky. To do so, we investigate the following aspects regarding \textit{the entire dataset}.

We first report general information about the prevalence of flaky tests and fault-revealing tests in order to have a better view of the failures occurring in each build. Then, we report the number of fault-revealing tests also found as flaky by the Chromium CI (reruns) in other builds (we further refer to them as fault-revealing flaky tests). Finally, we also check for the number of failing builds that only contain fault-revealing flaky tests. We consider these builds important since the related faults are not detected by any non-flaky test and would be missed if flaky test detectors were used.

% Figure Tests per Build
\begin{figure}[!htbp]
\centering
\includegraphics[width=0.8\textwidth]{figures/chromium/testsPerBuild.png}
\caption{Number of flaky tests and fault-revealing tests per build. On average, there are 250 flaky tests per build and 1 fault-revealing test per failing build.}
\label{fig:testsPerBuild}
\end{figure}

Figure~\ref{fig:testsPerBuild} shows the distribution of flaky tests and fault-revealing tests in the studied builds. We observe that there is an average of 178 flaky tests per build with a low standard deviation (41), showing that flakiness is prevalent in the Chromium CI. In the case of fault-revealing tests, taking into account all builds would result in an average number of tests close to 0 as a majority of builds are exempt from them. Thus, for better visualisation, we only considered builds containing at least one fault-revealing test (\ie failing builds). The average number of fault-revealing tests per failing build is 2.7.
%, which is why we dot see the lower quartile in the Figure. 
The standard deviation for fault-revealing tests is 14.9 and the number of fault-revealing tests reported in one build goes up to 579 in our dataset.

\begin{table}[ht]
\caption{Number of builds containing each studied test type. All builds contain flaky tests. \nicefrac{1}{4} contain fault-revealing tests. Among the failing builds, \nicefrac{3}{4} contain only fault-revealing tests that are flaky in other builds.}
\label{table:DiscBuilds}
\centering
\begin{tabular}{c|c} 
 \toprule
 \textbf{Builds containing} & \textbf{Number} \\ [0.5ex] 
 \midrule
 Flaky tests & 10,000 \\ 
 Fault-revealing tests & 2,415 \\ 
 Fault-revealing flaky tests & 1,974 \\ 
 Exclusively fault-revealing flaky tests & 1,766 \\ 
 \bottomrule
\end{tabular}
\vspace{-1em}
\end{table}

Table~\ref{table:DiscBuilds} provides, for each type of test, the number of builds that contain at least one instance of this type. We note that all builds contain at least one flaky test (a test that flaked during this build). In Chromium CI, flaky tests are non-blocking and will not cause a build failure. That is, tests flaking within the build are ignored during this build. 

Developers are expected to investigate test failures only when they occur consistently across 5 reruns (resulting in a fault-revealing test). Such fault-revealing tests occur in 24.15\% of the builds (i.e. in 2,415 builds). Interestingly, 1,974 of these builds (i.e. 81.73\%) contain fault-revealing tests that flaked in previous builds, indicating that \emph{tests with a flake history should not be ignored in future builds}. Perhaps worse, in 1,766 builds \emph{all} fault-revealing tests have flaked in some previous builds, indicating that no "reliable" tests identified the fault(s).

% \begin{table}[ht]
% \caption{Information about tests. 22,477 tests are exclusively flaky among all builds. 2,343 tests are fault-revealing, among which \nicefrac{1}{3} are flaky in other builds.}
% \label{table:rq1tests}
% \begin{center}
% \begin{tabular}{|c|c|c|} 
%  \hline
%   & \textbf{Number} & \textbf{Ratio} \\ [0.5ex] 
%  \hline
%  Tests & 209,530 & 100\% \\ 
%  Passing tests & 198,273 & 94.6\% \\ 
%  Exclusively passing tests & 184,710 & 88.2\% \\ 
%  Flaky or Fault-revealing tests & 24,820 & 11.8\% \\ 
%  Exclusively flaky tests & 22,477 & 10.7\% \\ 
%  Exclusively fault-revealing tests & 1,446 & 0.7\% \\ 
%  Fault-revealing flaky tests & 897 & 0.4\% \\ 
%  \hline
% \end{tabular}
% \end{center}
% \end{table}

% Figure Venn diagram
\begin{figure}[!htbp]
\centering
%\vspace{-0.5m}
\includegraphics[width=0.8\textwidth]{figures/chromium/venn.png}
\vspace{-1em}
\caption{Distribution of tests in our dataset. 22,477 tests are exclusively flaky among all builds. 2,343 tests are fault-revealing, among which \nicefrac{1}{3} are flaky in other builds.}
\label{fig:venn}
\vspace{-0.7em}
\end{figure}

By investigating the status of all tests across all builds -- see Figure~\ref{fig:venn}. Among the 209,530 tests of the Chromium project, 24,820 have failed in at least one build, including 22,477 that were always flaky. Thus, 2,343 tests were fault-revealing in at least one build, i.e., they attested the presence of faults, 897 were also flaky in at least one other build. That is, \emph{38.3\% of tests that have been useful to detect faults have also a history of flakiness}.

%'We visually represent the set of flaky tests and fault-revealing tests using the Venn diagram in Figure~\ref{fig:venn}. Across the 10,000 builds in our dataset, we find 2,343 fault-revealing tests. Among them, 1,446 are exclusively fault-revealing, \ie they were never found to be flaky in other builds. On the contrary, 897 fault-revealing tests were marked as flaky in at least one other build. This represents \nicefrac{1}{3} of fault-revealing tests.
%% that most of the 209,530 tests in our dataset are passing tests with 88.2\% having not failed once over the 10,000 builds we collected. Regarding failures, 24,820 tests were found to be flaky or failing in one or more builds. Among them, a majority, \ie 90\%, are exclusively flaky.  Interestingly, 2,343 tests are fault-revealing tests, \ie 1.1\%. Among them, \nicefrac{1}{3} exhibit both flaky and failing behaviour as they were reported as flaky in one or more other builds. \\

\begin{tcolorbox}[
    left=2pt,right=2pt,top=2pt,bottom=2pt, %margin  
    arc=0pt, % corners
    boxrule=1.2pt % line width
]
Flakiness affects all Chromium CI builds and mixes critical (fault-revealing) signals with false (flakiness) signals. Indeed, 81.7\% of builds contain fault-revealing tests that were flaky in some previous builds, and 38.3\% of all tests flake in some builds and reveal faults in other builds. 
%\textbf{RQ1:} In the Chromium CI, flaky tests are frequent in all builds. Critically, an important \nicefrac{1}{3} of regressions are reported by fault-revealing flaky tests. In addition, almost \nicefrac{3}{4} of failing builds exclusively contain regressions found by flaky tests (fault-revealing flaky tests). 
\end{tcolorbox}
\section{Threats to Validity}
\label{sec:chromium-threats}

\subsection{Internal Validity}
The main threat to the internal validity of our study lies in the use of vocabulary-based approaches as predictors of flaky tests and failures. Approaches leveraging other features, \ie dynamic, static, or both, could perform differently. As explained in section ~\ref{sec:evaluationSelection}, many features are difficult to extract in the case of Chromium (\eg test smells or test dependency graphs) and features relying on code coverage are not considered due to the overheads they introduce and the difficulty of instrumenting the entire codebase. Although this limits our feature set, the same situation appears in many major companies such as Google and Facebook. Nevertheless, our key insight is that many regression faults are discovered by flaky tests, meaning that they would have been missed even by any flaky test detector that correctly considers them as flaky. 

As seen in previous sections, most of the research on flakiness prediction focuses on classifying tests. Although in this paper we highlight the need for ---and focus on--- detecting failures, one may wonder what would be the performance of the studied techniques when aiming at detecting flaky tests (instead of flaky test failures). To this end, we trained a model using our dataset to distinguish flaky from non-flaky tests and found similar results with those reported by the literature, i.e., MCC 0.77 when shuffling data and MCC 0.52 when performing a time-sensitive evaluation. The above result shows that the problem of targeting flaky tests is easier and more predictable. However, as shown by our analysis it is misleading as more than 2/3 of the regression faults are missed by such methods. 

\subsection{External Validity}
We show that detecting flaky tests (instead of failures) is harmful as it can miss many regression faults. This is the case for the Chromium project and, while we believe Chromium to be representative of other software systems, we cannot guarantee that findings would generalise to other projects. Similarly, the performance of the different models we report may vary depending on the project. Here, we mainly focus on web/GUI tests and flakiness might have different causes in HTML and Javascript testing compared to other programming languages. 

Nevertheless, we believe that flaky tests are useful since developers tend to keep them instead of discarding them. Therefore, flaky test signals should not always be considered as false. 

\subsection{Construct Validity}
We assume that all fault-revealing tests in our dataset indeed reveal one or several issues in the code. This is the information reported by the Chromium CI as of today. It is possible that some fault-revealing tests are actually flaky tests as they might not be executed in a sufficient amount of time. However, reruns cannot guarantee that a test is not flaky. As this information is currently used by Chromium's developers and further verification is non-trivial, we rely on it as ground truth for our dataset. Passing tests used as non-flaky tests could also be mislabeled in our dataset. Though, there is a consequent number of passing tests and it is unlikely that many would actually be flaky. Furthermore, to strengthen our confidence in our set of non-flaky tests, we remove from the set of passing tests any tests that were found to be either flaky or fault-revealing in any of the 10,000 builds.

Additionally, it is possible that more than one regression fault is present in the case of several fault-revealing tests that fail in one build. Although this could alter the results we report in RQ1, it would actually strengthen our key message as even more faults could have been missed.
\section{Conclusion}
\label{sec:chromium-conclusion}

In this paper, we investigated the utility of existing vocabulary-based flaky test prediction methods in the context of a continuous integration pipeline. To do so, we collected data about 23,374 flaky tests and 2,343 fault-revealing tests composing a dataset of 1.8 million test failures representing the actual development process of more than 10,000 builds corresponding to a period of 9 months. Thus, we empirically evaluated the prediction methods and found similar performance compared to previous studies in terms of precision and recall. Despite the (very) high accuracy to detect flaky test failures, we also found that 76.2\% of fault-triggering test failures were misclassified as flaky by the prediction methods, indicating major losses on the fault revelation capabilities of the test suites. Going a step further, we also showed that flaky tests have a strong ability to detect faults, with \nicefrac{1}{3} of all regression faults being revealed by tests that have experienced flaky behaviour at some point in the lifetime of the project under analysis.

These findings motivated the need for failure-focused prediction methods. To this end, we extended our analysis by checking the performance of failure-focused models (trained on failures instead of tests) and found that they result in similar accuracy and fewer false positives. We also found that considering test execution features such as the run duration and the historical flake rate was helpful to increase its ability to discern flaky failures and fault-triggering failures. However, our results still miss a large number of test failures, 42.3\%. Therefore, we believe that the current performance is not actionable and that additional research is needed in order to tackle this vastly ignored problem of flaky test failure prediction over flaky tests.

Our future research agenda aims at improving the performance of flaky test failure detection techniques by using additional features and artificial data (augmenting the training data with new positive examples to tackle the class imbalance issues). We also plan to develop failure-cause interpretations for the techniques so that they could be usable by the Chromium developers.  