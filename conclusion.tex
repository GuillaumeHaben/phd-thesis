\chapter{Conclusion}
\label{chap:conclusion}

\chapterPage{
This chapter presents the overall conclusion of the dissertation and proposes potential research directions. 
}

\section{Summary of contributions}

In this dissertation, we presented studies, techniques and tools that contribute to reducing the maintenance cost that is associated with the evolution of \gls{suit}s. Indeed, with the increased adoption of Agile and DevOps methodologies, quick and reliable test feedback on every code change becomes essential, leading SUITs to evolve continuously alongside the system they exercise. However,  the user interface is a part of the system that tends to undergo rapid evolutions, causing \gls{suit}s to be particularly prompt to break. In this context, this dissertation brings the following contribution: (1) an empirical study that evaluates how \gls{suit}s evolve in a large industrial system; (2) an empirical study that measure the prevalence of bad design practices potentially contributing to test fragility and their higher maintenance cost; and (3) a novel approach to render \gls{suit}s more resistant to \gls{sut} evolution by creating more robust locators.

The goal of the first contribution is to shed light on how and why \gls{suit}s undergo heavy maintenance. Our analysis reveals that test fragility (the sensitivity to \gls{sut} evolution) and test clones (keywords with similar test functionality) are the most important problems of \gls{kdt}. On the positive side, our study provides evidence that following the good design practices of \gls{kdt} (such as  the  separation of concern and the reusability of keywords) has the potential to reduce the required number of maintenance changes. Our analysis shows that this reduction is approximately 70\% demonstrating major benefits of \gls{kdt}. Our results also help improving the understanding on the fine-grained changes performed during the evolution of \gls{kdt}.  We provide a taxonomy of test code changes and reveal the presence of test clones caused by the difficulty of selecting appropriate keywords. We believe that the main drawback of \gls{kdt} lies in the absence of appropriate tooling, allowing to deal with test code growth, navigation and comprehension. To address this concern, we introduce \tool\, an automated approach to provide testers with visibility on their test code base. Finally, we show that practitioners agree with the promises of \gls{kdt}, on the advantages of the separation of concerns and the reusability of keywords. 

%Furthermore, we observe that test fragility causes a constant test adaptation (even in response to simple \gls{gui} changes) and that test clones are prominent. We show that over 90\% of the project keywords are changed and over 30\% of the the keywords are clones. We also find that among identical clones, 50\% of them co-evolve. These findings indicate the need for automated test repair and refactoring techniques.  

With the second contribution, our objectives were to better understand which sub-optimal decision within the test code could have an effect on the maintenance by measuring the diffusion  bad practices and their refactoring from the test code base. To achieve these goals, we combine a multi-vocal literature review and an empirical study on a large industrial project and 12 open-source repositories.
Relying on the multivocal literature review, we build a catalogue of 35 \gls{suit} smells. For 12 out of 35 of the smells from this catalogue, we propose an automated approach for detecting the introduction and refactoring of these \gls{suit} smells. Leveraging this  approach, we evaluate the prevalence of \gls{suit} smells and their refactoring in our industrial and open-source projects. Even though our empirical results suggest that most tests present the symptoms of bad practices, less than half of them ever experience refactoring during their lifespan. However, while refactoring actions are rare, \gls{suit} smells like \emph{Narcissistic} and \emph{Middle Man} still disappear from the code base as a side effect of unintended maintenance and removal of symptomatic tests and the apparition of clean ones.

Finally, the third contribution tackles the breakages caused by second-generation locators. Because they rely heavily on internal properties of the elements they visit, second-generation locators can be problematic in the case of automated \gls{gui} testing as it leaks structural details of the \gls{sut} that should not be present in such tests. We propose HPath, a second-generation locators inspired by third-generation locators which avoid exposing internal details to the test by relying only on the rendered properties of the \gls{sut}. The practical benefits of HPath can be measured via its capability to generate more flexible locators than the current second-generation techniques. The results from our experimentation suggest that HPath is able to reduce test breakages compared to classical approaches relying on internal properties. However, while the results are promising, for HPath to be useful, when developing the user interface, developers need to rely on modern implementation of the \gls{html} standard and follow good practices.

\section{Perspectives}

In the following, we discuss potential future research that follow on the contributions and ideas presented in this dissertation:

\begin{itemize}
    \item \textbf{Effects of \gls{suit} smells:} In Chapter~\ref{chap:smells-system-user-interactive-test}, we present an exploratory analysis of the diffusion and the refactoring of bad practices in \gls{suit}s. In an attempt to better under their impact and propose solutions to avoid it, we conduct a deeper analysis of the effects associated with two of \gls{suit} smells: code duplication (Section~\ref{sec:evolution-results-rq3}) and fragile locator representation (Chapter~\ref{chap:robust-locators}). However, we believe that more research need to be conducted to further our understanding of the potential impact of the \gls{suit} smells presented in Section~\ref{sec:results-smells-catalog}.
    
    \item \textbf{Improvements on locators representation:} In Chapter~\ref{chap:robust-locators}, we propose an approach to exploit properties that make third-generation locators more robust to change and to integrate them in second-generation locators. Unfortunately, many current framework do not allow for the extraction of the representation layer in a way that allow to rely on second-generation locators. Building tooling to automatically extract a tree representation of a \gls{gui} using computer vision would allow to make our work to converge second- and third-generation locators generalizable to any \gls{gui}-based application.
    
    \item \textbf{Test execution analysis:} In this dissertation, we discuss noise introduce in the signal provided by test failures that is attributed to test breakage. However, there exist another major source of noise impacting the usage of \gls{suit} at scale, test flakiness. Indeed, by construction, \gls{suit}s rely on the system as a whole and operate asynchronously from the \gls{sut}. Therefore, they are more sensitive to  variations and non-determinism occurring either in the \gls{sut} or the infrastructure it relies on. Moreover, even when the failure is revealing a fault in the \gls{sut}, because of the size of the test, isolating the sub-system responsible for the failure becomes challenging. Consequently, we advocate for more research on the automatic analysis of test failures as the signal provided by the failure itself remains low. 
\end{itemize}
