\section{Threats to Validity}

Threats to the external validity result from the generalization of our results outside the context of the study. Conducting the study with one industrial partner, the conclusions we draw may not be able to generalize to other companies using \gls{kdt}. However, \emph{SubjectA} is built using popular technologies, i.e., web frameworks and Java services, which are wide-spread across the industry. Secondly and most important, this study is the first one, to the best of our knowledge, that analyzes the evolution of \gls{kdt} test suites based on real-world data. Of course, this does not preclude the need for other studies to investigate further our results. Another potential threat to generalization originates from the fact that we interviewed only 3 testers for RQ4. 

Threats to the internal validity are due to the design of the study, potentially impacting our conclusions. The simple syntax of the test code allows for a robust model to be constructed. Our change algorithm presents some limitations: although phase 2 is based on the state of the art, it cannot detect \emph{Move} operations, resulting instead in two operations \emph{Delete} followed by an \emph{Insert}. This limitation might have influenced our results during the accounting of the number of changes. However, the rather low number of the \emph{delete step} operations (Table~\ref{table:total_changes}) indicates that this effect is marginal. Regarding the clone detection algorithm, as shown in \cite{Roy2009}, the rate of false-positives is known to be low for Type I and Type II clones.

Threat to construct validity result from the non suitability of the metrics used to evaluate the results. The main threat lies in the division of our work in two periods: ``Creation'' and ``Maintenance''. While empirical data motivated this separation, they lack of theoretical grounding. Further work on the test execution is needed to better motivate this decision.
