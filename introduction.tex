\chapter{Introduction}
\label{chap:introduction}

\chapterPage{
With GUI-based applications becoming ubiquitous as a mean for users to interact with software, the question of their correctness is of utmost importance. Yet, testing such applications still poses major challenges to practitioners. 
After introducing four approaches employed when creating \gls{suit}, we introduce the challenges that come with them.
Before describing the three contributions of this dissertation, we scope the addressed sub-challenges tackled.
}

Today, companies need to produce software faster than ever in order to remain competitive. To do so, they rely mainly on two different strategies: the integration of a continuous deployment pipeline and the establishment of agile methodologies within the company. These strategies allow software producers to increase the release rate while maintaining a high quality for the software that is shipped.

One of the keystones of the quality process is based on the adoption of strategies to assess that both functional and technical requirements are met. Because of the complex dynamic runtime of most modern applications, static analysis techniques face limitations in revealing violations of the requirements. Thus, most teams heavily rely on testing to reveal faults in their software. 

We define testing as a technical investigation to assess the quality of a \gls{sut} by executing the latter against a set of input values. Tests can further be classified through the scope they target into three main categories: unit, integration, and system tests. Unit testing covers the tests that are testing a unit in isolation, be it a class, a function, or any other software component,  to validate each unit individually. In integration testing, related units are grouped together and tested as a group. Finally, system testing covers the tests evaluating the system as a whole to ensure its compliance with its requirements.

Because unit and integration tests target more restricted portions of the application, they offer better control to the tester who can isolate functionalities from its environment and tests specific behaviors of the system sub-parts. On the other hand, system tests, and especially those interacting with the user interface, target the system as a whole (usually as a black box). Therefore, they are sensitive to any change in the \gls{sut} and its interface. This property makes system tests more fragile, \ie\ they may break as a result of non-functional changes that are not targeted by the test.

In this work, we analyze the evolution of those system tests interacting with the user interface to shed light on the reasons why these tests require costly maintenance \cite{Alegroth2016, Coppola2020}. Then, we propose some solutions as to how to create more robust test suites, \ie test suites that are less likely to break following non-functional evolutions of the \gls{sut}.

\section{GUI Testing}
\label{sec:introduction-gui-testing}

User Interfaces are designed to provide a mean for the user to interact with an application. Unfortunately, as interfaces become more user-friendly, the underlying technology becomes more complex \cite{Myers1994}. Indeed, as depicted by \textcite{Myers1995}, user interfaces require the system to  interact with complex graphical components, to provide multiple ways to provide the same command, to manage multiple asynchronous input devices through the use of asynchronous event loops, or to provide an interactive computer programming environment (\eg\ read–evaluate–print loop). To ease development, the literature introduces different architectural designs to decouple the code responsible for the user interaction from the rest of the system such as the Entity-Component-System model \cite{Raffaillac2018} or the Model-View-Controller \cite{Krasner1988}.

Nevertheless, while these architectural solutions offer advantages by providing better isolation when testing individual components, in the case of system testing, the challenges remain open. Indeed, as the complexity of user interfaces and their associated code continues to grow, testing through the user interface for functional correctness remains challenging, but vital to help ensure the safety, robustness, and usability of the \gls{sut}.

Furthermore, today, \gls{gui} are becoming ubiquitous as a mean for users to interact with a software application \cite{Myers1992, Myers1995, Brooks2009, Memon2010}. In this work, we define a \gls{gui} as a hierarchical, graphical front-end to a software system that accepts as input user-generated and system-generated events from a fixed set of events that result in graphical output \cite{Memon2007} and/or alter the state of the software \cite{Nguyen2014}. A \gls{gui}-based application is a software for which the \gls{gui} is the main mode of interaction to the detriment of other types of user interfaces. As a consequence of the predominance of \gls{gui}-based applications, a lot of effort has been geared towards \gls{gui} automation, notably focusing on desktop applications \cite{Nguyen2014, Advolodkin2018, Pezze2018}, web applications \cite{Mesbah2009, Biagiola2019}, mobile applications \cite{Machiry2013, Gomez2013, Mao2016, Salihu2019, Yu2019} or cross-platform applications \cite{Canny2020}. However, as we show later in this section, major challenges remain open.

The question of the correctness of a \gls{gui}-based application is of utmost importance since prototyping and developing the behavior of a system is harder than prototyping and developing the appearance of the application \cite{Myers2008}. It is in this context that we introduce \gls{gui} testing which encompasses the techniques used to test user interface layout and its behavior. More precisely, \gls{gui} testing can be defined as ``the process of testing a software application with a graphical front-end to guarantee that it meets its specification by performing a sequence of events against the \gls{gui} elements'' \cite{Cunha2010, Banerjee2013, Issa2012}. Thus, relying only on stimuli inputs and output reactions, a \gls{gui} test is considered as a black-box test. Additionally, we define a \gls{suit} as an automated \gls{gui} test that exercises the \gls{sut} as a whole (system level). Note that in this work focusing on automated system-level test interacting with the \gls{sut} through its \gls{gui}, we use \gls{gui} test and \gls{suit} interchangeably.

Strategies have been developed to exercise the user interface to ensure that functional requirements are satisfied. Different techniques have been proposed but all rely on the same principle. A test framework executes events belonging to \gls{gui} components and monitors the resulting changes to the program state \cite{Nguyen2014}. In other words, each technique generates a sequence of test step, that will be executed against the \gls{sut} (input) and captures some indication as to what is the current state of the system (output). Each test step consists of a triple: (1) an action to be performed, (2) the \gls{gui} element on which to perform the action, and (3) an optional value passed to the element. Consequently, \gls{gui} test generation can be classified using the way the sequence of test steps is generated. To do so, we use existing classification frameworks proposed by the literature \cite{Anand2013, Amalfitano2017} to reframe them in the context of \gls{gui} test generation and its maintenance.

In the remainder of this section, we discuss current techniques used to generate \gls{suit}s. The techniques are ordered based on their theoretical potential for automation during the test generation process.

\subsection{Random GUI Testing}
\label{sec:introduction-random-gui-testing}

In random \gls{gui} testing, also known as monkey testing or random-walk, tests are automatically created by generating sequences of events at random in the hope of exploring the \gls{sut} to reveal failures. Because no requirements are passed as input such techniques rely on specific constraints that are application or domain invariant oracles \cite{Mesbah2009} to assess correctness. Thus, any sequence of events that causes the \gls{sut} to violate an invariant is considered as a fault revealing test case \cite{Barr2015}. These properties make random \gls{gui} testing cheap to run on a large number of versions/configurations of applications.

Typically, such tools rely on the concept of fuzz testing which performs arbitrary or contextual events against the SUT. Tools like Monkey \cite{Google2020},  ATUSA \cite{Mesbah2012} or Dynodroid \cite{Machiry2013} rely on pseudo-random input events which include both user interface events and system events and assess the generated sequences against invariant oracles \cite{Amalfitano2011}. 

One of the major challenges encountered when relying on random \gls{gui} testing resides in the exploration of the application states. Indeed, with modern applications, the infinite number of combinations of user inputs leads to a potentially large and sometimes infinite number of reachable states. Consequently, the search space cannot be covered exhaustively leading to potentially low state coverage \cite{Canny2019}. Furthermore, in the case of more complex input, such as a chain of character, randomly generated input have often low chances of producing feasible test cases, thus requiring a large number of input generations to find interesting test cases (\eg if a login page exists, only a combination of a valid user name and password would allow the test case to continue its exploration behind the login wall).

To alleviate these limitations, different approaches emerged introducing algorithms relying on heuristics to better explore the application state space and consequently generate more fault-revealing test cases. For example, EXSYST \cite{Gross2012}, EvoDroid \cite{Mahmood2014} and Sapienz \cite{Mao2016} propose to use search-based algorithms to explore the space of possible sequences and generate tests for Android applications while Dynodroid \cite{Machiry2013} exploits machine learning algorithms to achieve the same goal. The goal of the algorithm is to generate short sequences of input while improving coverage (code \cite{Gross2012} and/or state \cite{Machiry2013} coverage) to produce fault-revealing sequences. More recently, tools such as DIG \cite{Biagiola2019} try to increase the coverage by specifically targeting the input generation problem. Finally, we can mention which rely on behavioral invariants (\eg ALARic \cite{Riccio2018} which relies on the Android Activity lifecycle) to explore the space and find faults. While these approaches look promising, today, monkey testing requires the user to provide some inputs (\eg login and passwords) to allow the algorithms to explore more states and only manages to partially cover the search spaces.

To summarize, random \gls{gui} testing offers a cheap alternative to classical test scripting which can be useful to detect crashes and other invariants in an exploratory fashion. Nevertheless, because of the lack of knowledge of the functional requirements and the lifecycle of the application, exploring the application and covering hidden behaviors remains challenging. Thus, while companies do use random \gls{gui} testing, they use it in addition to other techniques that we present below.

\subsection{Model-Based Testing}
\label{sec:introduction-model-based-testing}

\gls{mbt} is an approach encompassing the processes and techniques for the automatic derivation of abstract test cases from abstract models \cite{Utting2012}. Functional specifications of a system, with the assumption that they are precise enough, are used as input to design process models which in turn are used to generate automated test cases \cite{Gupta2011}. This process can be devised in the five below steps \cite{Utting2012}:

\begin{enumerate}
    \item A model of the \gls{sut} is built from informal requirements or existing specification documents;
    \item Selection criteria are chosen, to guide the automatic test generation;
    \item Test selection criteria are transformed into test case specifications;
    \item A set of test cases is generated with the aim of satisfying all the test case specifications;
    \item Finally, the test cases are executed against the \gls{sut}.
\end{enumerate}

Unfortunately, most of the current \gls{gui}-based applications are not derived from a model, thus, cannot rely on an \emph{a priori} model encompassing all the expected behaviors of the application. Besides, even though advances have been made in Model-Based Software Engineering, complex \gls{gui} applications exhibit behaviors that cannot be expressed by models found in \gls{gui} testing \cite{Lelli2015}. This limitation causes \gls{mbt} to offer limited applicability which explains why practitioners shy away from the model-based solutions proposed in the scientific literature.

To tackle this shortcoming, researchers propose methods to automatically derive a model of the application by reverse-engineering \cite{DiLucca2002, Memon2003, Amalfitano2015, Canny2020}. This field, also referred to as Model Learning, relies similar techniques to the ones presented in Section~\ref{sec:introduction-random-gui-testing}, when discussing Random \gls{gui} exploration. For example, \textcite{Memon2007} introduces an event-flow model to formally represent events and event interactions in a \gls{gui}-based application. However, because of the large space of events and interactions present in a given application, manually building the model is prohibitive. Thus, the author proposes to rely on reverse-engineering to automatically derive the models by automatically exploring applications (in a similar fashion as random \gls{gui} testing) in a process called \gls{gui} ripping \cite{Memon2003} using tools such as AndroidRipper \cite{Amalfitano2012}. Unfortunately, models built that way do not guarantee completeness, leading to difficulties such as assessing if the coverage offered by the model is sufficient or choosing inputs during exploration. While the literature offers partial solutions such as improving the input generation \cite{Biagiola2019} by eventually relying on coverage \cite{Yuan2007}, or new ways to derive models from the \gls{gui} and its requirements \cite{Canny2020} the question of automatic exploration of modern \gls{gui}-based application remains an open problem.

\subsection{Record \& Replay}
\label{sec:introduction-record-and-replay}

This leads us to the third category of \gls{gui} testing, Record \& Replay which generates test cases from sequences of user interactions provided by the tester and not by exploring the model in a random fashion (Random \gls{gui} Testing) or by relying on a model (Model-Based Testing). As its name suggests, Record \& Replay works in two phases: a record phase and a replay phase:

\begin{enumerate}
    \item During the record phase, the tester manually interacts with the application, thereby generating events on the \gls{sut}. The tool records the interactions and a part of the \gls{sut} response state as specified by the tester.
    \item During the replay phase, the recorded test cases can be replayed on subsequent versions of the \gls{sut} and the captured state of the application are used as test oracles.
\end{enumerate}

The generated test cases require human intervention during the capture phase and can be rerun, reducing the overall effort of regression testing. Furthermore, because the test cases are generated by the framework, no particular skills are needed from the tester. \textcite{DiMartino2021} show that even when human testers have limited information about the system, they achieve better coverage metrics than automatic test generation techniques when using Record \& Replay.

Despite its advantages when generating the test cases, the generated tests tend to be fragile \cite{Hammoudi2016}. Indeed, whenever the application naturally evolves, the generated test cases might break, requiring testers to regenerate the test. This limitation can be exacerbated by agile and continuous delivery practices where applications and their user interfaces are under constant evolution.

To address this drawback, the research community proposes different ways to tackle the problem of fragility. Some attempts have been geared towards generating more robust test cases \cite{Hammoudi2016b}. Other researchers focus their attention on the more efficient way to record the interaction \cite{Ronsse1999}. Finally, instead of recording the interactions as sequences of actions, tools such as EventFlowSlicer \cite{Saddler2017} generate a model from these interactions. Thus, similarly to \gls{mbt}, subsequent versions are tested against the model. \textcite{Amalfitano2019} take the idea of relying on \gls{mbt} further by allowing the tool to perform exploration of the \gls{gui}.

Therefore, we see that while the generation of test cases using Capture \& Replay is efficient, the approach suffers from drawbacks in terms of the maintainability of the generated test suite. This issue is all the more relevant as the generated tests are often obscure, lacking any type of internal hierarchy, thus, compromising any attempt to manually fix such tests. 

\subsection{Test Scripting}
\label{sec:introduction-test-scripting}

Finally, the last technique covered in the section covers tests that are manually crafted by testers themselves. Following the general definition, a test script can be defined as a piece of executable code that executes a test case against the \gls{sut} and generates a verdict. In this work, we focus on test scripts interacting with the \gls{sut} through their \gls{gui}, namely, \gls{suit} scripts. \gls{suit} scripts aim at separating test design from the technical implementation of the tests.

Indeed, \gls{suit} scripts are typically used by teams performing acceptance testing. Acceptance tests ensure that a specific acceptance criterion, which can be functional or non-functional is met \cite{Pandit2015}. Conforming to the acceptance criteria both verifies that the \gls{sut} delivers the business value expected by the customer and guards against regressions or defects that break preexisting functions of the \gls{sut} \cite{Humble2010}. Hence, acceptance tests are not concerned with the internal implementation of the \gls{sut}, but with its overall behavior. Additionally, because such tests are business-facing, they are involved in the discussion between testers, developers, and business analysts, and, as such, should be readable by all stakeholders.

Practitioners propose design patterns to separate different concerns into different abstraction layers. For example, \textcite{Humble2010} propose the three following layers adopted by open-source and commercial tools like Robot Framework \cite{RobotFramework2020}, Cucumber, JBehave, Finess, or TestComplete:

\begin{itemize}
    \item \textbf{Acceptance Criteria.} It contains functional behavior, quality characteristics, scenarios, or business rules that the test is targeting. These are usually written in a form close to natural language. The goal of this layer is to express for all the stakeholders the requirements that are being automatically executed.
    
    \item \textbf{Test Implementation Layer.} It contains all the underlying implementation of the test called by the acceptance criteria layer. Because test implementations that refer directly to the application \gls{gui} elements are fragile, it is not uncommon to see large portions of acceptance test suites break when a single \gls{gui} element changes. While deferring the interactions with the application to the application driver layer does not make tests more robust to changes, it will make their maintenance easier with a better separation of concerns. Indeed, this layer follows the logic and vocabulary of the \gls{sut}, instead of the technical details of the implementation with its \gls{gui} elements.
    
    \item \textbf{Application Driver Layer.} It is the layer that understands how to interact with the \gls{sut}. While the two other layers use only the domain language of the business, the application driver layer is expressed in the domain language used by the drivers that communicate with the \gls{sut}. One of the consequences of a well-designed application driver layer is to improve test reliability. Moreover, because of the high degree of reuse that is implicit when relying on an application driver layer, complex interactions and operations can be written once and used in many tests.

\end{itemize}

One approach commonly used in industry that promotes this architecture is \gls{kdt}. Indeed, \gls{kdt} advocates that a separation of concerns allows tests to be written easier, helps to limit code duplication, and enables experts from different fields and backgrounds to work together, at different levels of abstraction, for the creation of the tests.

\begin{figure}
\centering
\caption{Example of Robot Framework test}
\label{fig:robot-script}
\begin{minipage}{0.8\linewidth}
\begin{lstlisting}[]
    <@\textcolor{block}{*** Settings ***}@>
    Test Teardown     Close All Browsers
    Library           Selenium2Library
    
    <@\textcolor{block}{*** Test Cases ***}@>
    <@\textcolor{keyword}{A user logs in with his username and password}@>
        Given browser is opened to login page
        When user "demo" logs in with password "mode"
        Then welcome page should be open
    
    <@\textcolor{block}{*** Keywords ***}@>
    <@\textcolor{keyword}{Browser is opened to login page}@>
        Open browser to login page
    
    <@\textcolor{keyword}{User "}@><@\textcolor{variable}{\$\{username\}}@><@\textcolor{keyword}{" logs in with password "}@><@\textcolor{variable}{\$\{password\}}@><@\textcolor{keyword}{"}@>
        Input Text    username_id    <@\textcolor{variable}{\$\{username\}}@>
        Input Text    password_id    <@\textcolor{variable}{\$\{password\}}@>
        Click Button    validate_id
    
    <@\textcolor{keyword}{Open Browser To Login Page}@>
        Open Browser    <@\textcolor{variable}{\$\{LOGIN URL\}}@>    <@\textcolor{variable}{\$\{BROWSER\}}@>
        Maximize Browser Window
        Set Selenium Speed    0
        Login Page Should Be Open        
    
    <@\textcolor{keyword}{Go To Login Page}@>
        Go To    <@\textcolor{variable}{\$\{LOGIN URL\}}@>
        Login Page Should Be Open
        
    <@\textcolor{keyword}{Welcome Page Should Be Open}@>
        Title Should Be    Welcome Page
        
    <@\textcolor{keyword}{Login Page Should Be Open}@>
        Title Should Be    Login Page
    
    <@\textcolor{block}{*** Variables ***}@>
        <@\textcolor{variable}{\$\{SERVER\}}@>           localhost:7272
        <@\textcolor{variable}{\$\{BROWSER\}}@>          Chrome        
        <@\textcolor{variable}{\$\{LOGIN URL\}}@>        http://<@\textcolor{variable}{\$\{SERVER\}}@>
\end{lstlisting}
\end{minipage}
\end{figure}

To do so, \gls{kdt} \cite{Tang2008, Hametner2012} aims at separating test design from technical implementation. Its goal is to limit the exposure to unnecessary details and avoiding duplication. Figure~\ref{fig:robot-script} shows an example of a \gls{kdt} test. This test, named ``Valid Login'' (line 5, adopted from the official documentation of Robot Framework), is responsible for validating the correct behavior of the login form in an imaginary \gls{sut}. Lines 6--8 present the ``steps'' of the tests and, in \gls{kdt} parlance, they are calls to \emph{keywords}. In turn, these keywords are defined in the respective definition blocks between lines 10 and 28. Each keyword is itself decomposed into a series of steps. Keywords can have \emph{arguments}. For instance, keyword ``Open browser'' (line 12) takes two arguments, ``\$\{LOGIN URL\}'' and ``\$\{BROWSER\}''. The use of arguments to call keywords allows to further extend the reuse of keywords.

As can be seen from the figure, most parts of this fully automated test are written in plain English. This enables unobstructed collaboration between different experts while creating a test. For instance, a business analyst can write the high-level part of the test (lines 4--8) and an automation expert can implement the remaining part of the test (lines 10-35), adding the technical details to automate the steps.

However, while manually generated test scripts present advantages in terms of communication, compared to the other techniques presented, they present a major drawback: they are expensive to create and incur a high maintenance cost as the system they are testing naturally evolves, especially when good practices are not respected.

\section{Challenges of System User Interactive Tests}

As depicted in Section~\ref{sec:introduction-gui-testing} while different \gls{gui} testing techniques exist, they all come with advantages and limitations. When addressing the case of fully automated techniques such as random \gls{gui} testing and to a certain extent \gls{mbt}, exploring the \gls{gui} space remains an open challenge. Thus, practitioners while relying, on these approaches tend to complete the test suites with manual test generation.

Unfortunately, manual test generation comes with its challenges as well. Indeed, because the \gls{gui} layer exhibit a lot of variation across versions and configurations \cite{Gao2015}, \gls{suit}s  tend to be fragile, \ie breaking following non-functional changes resulting from the natural evolution of the \gls{sut}. Hence, test cases fail or require maintenance even though the specific functionalities they test remain unchanged \cite{Coppola2019, DiMartino2021}.

Better understanding why \gls{gui} test scripts break is crucial to understanding how to limit their fragility. The fragility of the test code is measured by its propensity to break following minor changes in the production code not affecting its functionalities \cite{Garousi2016, Coppola2019}. In the previous definition, by minor changes, we understand changes not presenting any trouble to a human. Moreover, we define a test breakage as the event that occurs when the test raises exceptions or errors that do not pertain to the presence of a bug \cite{Stocco2018}. For example, a test exercising the login behavior of a system should not be impacted by the color of the \texttt{Login} button or its internal \texttt{id}. 

Therefore, to create more robust tests, we first need to understand which are the properties that cause them to break, be it in their structure, the way they interact with the \gls{sut} or how they synchronize with it. By isolating these properties and the invariants present in the \gls{sut} onto which tests can rely upon research can suggest good practices and tooling to help increase test robustness. As such, finding those properties and invariants is a prerequisite if we want to be able to introduce the use of \gls{suit}s at scale in fast pace development processes.

\section{Overview of the Contribution and Organization of the Dissertation}

This section presents the contributions of this dissertation to address the aforementioned challenges encountered when writing \gls{suit} scripts as well as the organization of this dissertation. 

\subsection{Contributions}

Following are the contributions of this dissertation:

\begin{itemize}
    \item \textbf{An empirical study on the evolution of \gls{kdt} in an industrial context (Chapter~\ref{chap:evolution-system-user-interactive-test}).} We conduct an empirical study on the evolution of \gls{kdt} suites. Our aim is to investigate the problem of test maintenance, identify the benefits of \gls{kdt} and overall improve the understanding of test code evolution (at the acceptance testing level). This information supports the development of automatic techniques, such as test refactoring and repair, and allows future research on test fragility. To this end, we identify, collect and analyze test code changes across the evolution of industrial \gls{kdt} test suites for a period of eight months. We show that the problem of test maintenance is largely due to test fragility (most commonly performed changes are due to locator and synchronization issues) and test clones (over 30\% of keywords are duplicated). We also show that the test design of \gls{kdt} test suites has the potential for drastically reducing (approximately 70\%) the number of test code changes required to support software evolution when compared to Record \& Replay. To further validate our results, we interview testers from \BGL\ and report their perceptions on the advantages and challenges of keyword-driven testing. 
    
    \item \textbf{An empirical study on the diffusion and refactoring of test smells in \gls{kdt} (Chapter~\ref{chap:smells-system-user-interactive-test}).} We conduct an empirical study on the diffusion and refactoring action observed in \gls{kdt} suites. Following our analysis on test maintenance, we focus on bad test scripting practices, \gls{suit} smells, and measure how often they  are refactored. To this end, we collect and analyze \gls{suit} smells and their refactoring across over two million test instances of an industrial composed of 48 test suites and project and 12 open-source projects. We show that the same type of smells tends to appear in industrial and open-source projects, but the symptoms are not addressed in the same way. Some smells tend to appear in most tests with a diffusion of up to 90\%. Yet refactoring actions are much less frequent with less than 50\% of the affected tests ever undergoing a refactoring. Interestingly,  while refactoring actions are rare, some smells tend to disappear through the removal of old symptomatic tests and the introduction of new tests not presenting such symptoms.
    
    \item \textbf{An approach to generate more flexible locators (Chapter~\ref{chap:robust-locators}).} One of the main sources for test fragility is the evolution of \gls{gui} elements tests are interacting with. To reduce the impact of such evolution, we propose a novel approach, HPath, to provide more flexibility to \gls{dom}-based locators by exploiting the semantics offered in the \gls{html}5 standard. We evaluate our approach on web elements mined from the test suites of two large open-source projects. Our results show that shorter locators are not necessarily more robust to change than longer ones. However, relying on better properties of the \gls{gui} elements can significantly increase their robustness. As such, when \gls{html}5 semantics are present, HPath can exploit rendered properties of web elements to generate expressive locators for 73.35\% of them. This reduces locator breakages from 64.99\% when relying on implementation properties to 0.49\% with rendered properties.
    
    \item \textbf{Ikora: Continuous Inspection for \gls{kdt} (Chapter~\ref{chap:ikora-inspector}).} We introduce Ikora, an automated tool that statically analyzes Robot Framework test suites, enabling the continuous inspection of test codebase written using \gls{kdt}. Ikora targets code written in the Robot Framework syntax and has been successfully deployed at \BGL, detecting issues otherwise unknown to the automation testers, such as the presence of duplicated test code, dead test code, dependency issues among the tests, and bad design practices.
\end{itemize}

\subsection{Organization of the Dissertation}

In the remainder of this dissertation, Chapter~\ref{chap:related-work} presents the previous work that is related to the contributions presented in this dissertation. Chapter~\ref{chap:evolution-system-user-interactive-test} presents an empirical study that evaluates the evolution of \gls{suit} scripts written using \gls{kdt} in an industrial setting. Chapter~\ref{chap:smells-system-user-interactive-test} presents an empirical study on the diffusion and refactoring of smells present in \gls{kdt} scripts both in industrial settings and in open-source software. Chapter~\ref{chap:ikora-inspector} describes the tools and frameworks built as the result of this work to contribute to the advances in good testing practices when developing \gls{gui} test cases. Finally, Chapter~\ref{chap:conclusion} concludes this dissertation and presents some potential future research direction.