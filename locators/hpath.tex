\section{HPath}
\label{sec:hpath-hpath}

Relying on the HTML standard described in section~\ref{sec:hpath-introduction-HTML}, we proposed \textsc{HTML Path Language (HPath)} a new query language to locate nodes in an HTML document. As its name suggests, our approach is similar to the XPath specification but tailored for HTML documents. While the target is the same as the CSS Selector (both targeting HTML documents), our scope is different. Indeed, our query language aims at providing test automation engineers more expressive and more flexible locators for GUI-based testing. In other words, the goal of our approach is to provide locators that leak as little structural details as possible by relying on properties of the HTML nodes that are rendered on the page to perform the query.

The intuition behind HPath lies in the fact that, if only visible features are permitted in the query, then the resulting locators should be more resilient to iso-functional structural changes. As a consequence, our approach helps to reduce GUI-based test fragility \cite{Thummalapenta2013, Hammoudi2016}. Moreover, relying on rendered properties (such as the label of an input) makes the locator (and, by extension, the test script) easier to understand %by humans reading a test or observing its execution report, because 
as the reliance on structural properties has been minimized.

\begin{table}
\centering
\caption{BNF of HPath.}
\label{tab:hpath-hpath-grammar}
\begin{tabular}{>{\raggedright}m{0.6in}>{\raggedright}m{0.1in} >{\raggedright}m{1.9in}}
\toprule
\code{LocationPath} &\code{:=} &\code{RelLocationPath | '/' RelLocationPath?}\tabularnewline
\code{RelLocationPath} &\code{:=} &\code{'/' Step | RelLocationPath '/' Step}\tabularnewline
\code{Step} &\code{:=} &\code{NameTest Predicate?  | NodeType '(' ')'}\tabularnewline
\code{NameTest} & \code{:=} & \code{Literal}\tabularnewline
\code{Predicate} &\code{:=} &\code{'[' PredicateExpr ']'}\tabularnewline
\code{PredicateExpr} &\code{:=} &\code{Number | FunctionCall}\tabularnewline
\code{FunctionCall} &\code{:=} &\code{FunctionName '(' ')' '=' '"' Literal '"'}\tabularnewline
\code{FunctionName} & \code{:=} & \code{'label' | 'legend' | 'caption' | 'figcaption'}\tabularnewline
\code{NodeType} & \code{:=} & \code{'text'}\tabularnewline
\bottomrule
\end{tabular}
\end{table}

Table~\ref{tab:hpath-hpath-grammar} presents the grammar of HPath in Backus–Naur form (BNF). HPath is expressed as a location path composed of a series of steps (in a XPath fashion, yet using different predicates). Thus, HPath navigates in the HTML documents based on the properties of the nodes it traverses. Each step comprises a \emph{NameTest}, which defines the type (tag) of the context element associated to an optional \emph{Predicate} or a node $N_D \not\in E_D$. Note that only text nodes are supported since they are the only ones being rendered on the page. \emph{Predicate} returns a boolean value filtering out parts of the tree during path traversal. HPath accepts a textual input for the resolution of the predicate (\emph{Literal}) and returns whether or not the value matches the property of the context node. Our implementation is available as a Chrome Plugin\footnote{Available at \emph{link anonymized for double-blind review}} allowing to generate HPath through the browser.

In the remaining of this section we present the properties of HPath empowering it to generate more flexible location paths. We use the notation $E_{\texttt{type}}$ to describe a set of elements of type \texttt{type}, and $e_\texttt{type}$ to refer to a specific element of such a set.

\subsection{Node traversal}
\label{sec:hpath-hpath-node-traversal}

Not all nodes in an HTML document are used for rendering. Indeed, placeholders can be introduced in the DOM to make pages dynamic (animation, dragging, etc.) or to provide better readability for web robots (e.g., Search Engine Optimization). Thus, by definition, these nodes should not appear in the location path when creating flexible locators for GUI-base testing. 

Therefore, the node traversal is done on a rendering tree, generated by the HPath engine, where all elements not affecting the rendering flow are pruned out. Referring to the semantic offered by the HTML standard, two types of elements, $E_{span}$ and $E_{div}$, do not hold any intrinsic semantic by themselves and act as placeholders. Indeed, if no styling is applied to these elements, they are ignored by the rendering engine\cite{Grigorik2019}. Leveraging on this property, HPath ignores any $e_{span}$ or $e_{div}$ that will not affect the rendering flow.  

Since the locator is computed after all styling properties have been assigned to their corresponding elements, all the elements hold all the styling information needed to compute the rendering tree. Then, the algorithm relies on the browser engine for style sheets processing. 

\subsection{Predicates}
\label{sec:hpath-hpath-predicates}

While computing predicates or any other node filtering mechanism, DOM-based locators usually rely heavily on implementation-dependent properties of the elements they visit (\emph{e.g.} id or class). However, these attributes can change without affecting the functionality of a page or any of its visible parts. For example, modifying the id of a button, if not affected by a style rule, will not have any effect on its rendering. Thus, HPath defines a few selected predicates relying on rendered properties. During the predicate resolution, if the extracted text matches the value provided, it returns a true value, false otherwise. The predicates can take the following form:

\begin{itemize}
    \item \textbf{label} Most of the interactive elements $E_{interactive}$ can be associated with a label element, $E_{label}$ to provide a textual clue about their functionality (\emph{e.g.} Label "Name" associated to a textual input so the users know they have to type their name). The HTML standard defines two strategies to link those elements: (1) $e_{label}$ is a direct parent of $e_{interactive}$ or (2) $e_{label}$ is a sibling of $e_{interactive}$ where the value of the attribute \emph{for} of $e_{label}$ equals the value of the attribute \emph{id} of $e_{input}$. The \textbf{label} predicate extracts the textual content (see Section~\ref{sec:hpath-hpath-text-extraction}) of $e_{label}$ associated with the context $e_{interactive}$. 
    
    \item \textbf{legend} Related children elements of a form element ($E_{form}$) can be grouped together through the use of fieldset element ($E_{fieldset}$). Any $e_{fieldset} \in E_{fieldset}$ can be associated with a textual value defined in a legend element, $e_{legend}$. According to the standard, if $e_{legend}$ is a direct child of $e_{fieldset}$, then it is associated to the parent $e_{fieldset}$. The \textbf{legend} predicate extracts the textual content of a $e_{legend}$ associated to the context $e_{fieldset}$. 
    
    \item \textbf{caption} Table elements, $E_{table}$, can be assigned a caption through the use of a caption element $E_{caption}$. According to the standard, if a $e_{caption}$ is a direct child of a $e_{table}$, then it is associated to the parent $e_{table}$. The \textbf{caption} predicate extracts the textual content of a $e_{caption}$ associated to the context $e_{table}$. 
    
    \item \textbf{figcaption} Figure elements, $E_{figure}$, can be assigned a caption through the use of a caption element $E_{figcaption}$. According to the standard, if $e_{figcaption}$ is a direct child of $e_{figure}$, then it is associated to the parent $e_{figure}$. The \textbf{figcaption} predicate extracts the textual content of $e_{figcaption}$ associated to the context $E_{figure}$. 
\end{itemize}

If no predicate can be generated to compute the location path, then the resulting path resembles the one that would be generated by the XPath algorithm, i.e., with only positioning predicate and no axis. The difference, however, is that HPath does not compute the path on the actual DOM but on the rendering tree (as described in Section~\ref{sec:hpath-hpath-node-traversal}).

\subsection{Content Extraction}
\label{sec:hpath-hpath-text-extraction}

In the current implementation of HPath, all predicates (except the positioning predicate) rely on textual information visible on the page. To extract the content of a text under an element, XPath simply extracts the content of the text node, $n_{text}$, which is a direct child of the context element. HPath goes beyond this procedure and analyzes styling information contained in the elements from the Inline Text Semantics category (see Table~\ref{tab:hpath-introduction-html5}). HPath takes advantages of this semantic and filters out any element from this category when extracting text. The resulting $n_{text}$ are then merged together to reconstruct the text that will be used by the predicates for evaluation. For example the "<b>Hello</b> world" is transformed to "Hello world".
