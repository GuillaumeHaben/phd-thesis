\section{Introduction}
\label{sec:hpath-introduction}

In Section~\ref{sec:introduction-gui-testing}, we define a \gls{gui}-based test as a sequence of test steps where each test step consists of a triple: an action to be performed, the \gls{gui} element on which to perform the action and an optional value passed to the element. Consequently, for each step, the test has to be able to uniquely identify the \gls{gui} element it has to interact with. The literature identifies three locator strategies, \emph{i.e.}, ways to locate \gls{gui} elements in order to interact with the \gls{sut} \cite{Bosch2014, Leotta2018}. Each locator strategy is referred to according to its generation, all raising different concerns in term of fragility to SUIT evolution. 

The first-generation locators are based on screen coordinates for interacting with the application. While this technique was good for Record \& Replay, the high maintenance cost of this approach drove away most practitioners to the benefit of the two next generations.

The second-generation locators, also known as property- or DOM-based locators, rely on properties of \gls{gui} components in order to locate them. Here, each page is associated to a \gls{dom} offering a tree-like structure that can be navigated by means of XPath queries or \gls{css} selectors. Exploiting the \gls{dom} structure instead of directly targeting the rendered page leads to locators not affected by the properties of the rendering artifacts, such as screen resolution. However, \gls{css} selector and XPath share one pitfall: they rely heavily on internal properties of the elements they visit (e.g., its attributes or its position in the \gls{dom} tree and hierarchy of tree nodes). Despite being a reasonable strategy for most tasks they are used for (e.g., information retrieval, content formatting), this approach can be problematic in the case of automated \gls{gui} testing where the reliance on these attributes leaks structural details of the page that should not be present in the tests. Previous research \cite{Thummalapenta2013, Hammoudi2016} has shown that this leakage of structural details makes tests more sensitive to changes not related to the behavior of the application, resulting in test breakage.

While second-generation locators are wildly adopted in industry, a third-generation offers promising potential \cite{Alegroth2015}. This third-generation locators, adopted in \gls{vgt}, uses computer vision to locate elements through the bitmap graphics shown to the user at runtime. This technique was born as an answer to the fragility of second-generation tools generating high maintenance costs under \gls{sut} evolution. Indeed, relying on computer vision allows third-generation locators to be more flexible under structural changes. However, while this technique is more adaptive to underlying structural changes than second-generation locators, minor changes in the \gls{gui} representation (\emph{e.g} minor changes in the layout or in the color of \gls{gui} components) might break tests relying on the exact representation of GUI elements. Furthermore, relying on computer vision to identify \gls{gui} elements remains a non-trivial task, even though considerable progress have been made, notably with the use of machine learning \cite{White2019}.

Hence, while second- and third-generation locators each offer their own advantages, they both still contribute to test fragility \cite{Aldalur2017, Alegroth2018}. As mentioned in Chapter~\ref{chap:related-work}, to overcome the test fragility problem, academia has investigated two families of approaches: generate tests more robust to \gls{sut} evolution \cite{Montoto2011, Thummalapenta2013, Leotta2014, Yandrapally2014, Leotta2015, Leotta2016, Zheng2018} or automatically repair tests following breakages \cite{Choudhary2011, Stocco2018, Kirinuki2019}. In this chapter, our contribution falls in the former line of research as we present an attempt at reducing test breakages resulting from \gls{sut} evolution. 

To this end, we propose a novel second-generation locator, HPath, which is similar to the XPath specification but exploits rendering properties unique to \gls{html} documents. The practical benefits of HPath can be measured via its capability to generate more robust locators than the current second-generation techniques. The idea behind it is to prune the \gls{dom} tree by computing which elements are rendered. Hence, while the algorithm works in a similar fashion as absolute XPath to traverse the tree, it works on a rendered tree instead of the original \gls{dom}. Furthermore, the remaining nodes only keep a subset of their properties, the ones that affect the display. Hence, by this tree transformation, HPath relies on similar attributes as third-generation locators (no reliance on structural details) while keeping the advantages of second-generation locators (tree representation which is easy to traverse).

Before presenting HPath, we formally introduce each of the elements from the presentation layer of a web application which are exploited by algorithm and current approach to traverse it.

\subsection{Document Object Model}
\label{sec:hpath-introduction-DOM}

The \gls{dom} is a standard maintained by the \gls{w3c} until 2004 and then took over by the \gls{whatwg}. The goal is to provide a common \gls{api} which can be used with different languages in broad range of environments to generate well-formed \gls{xml} documents. According to the definition provided by the \gls{w3c}, the \gls{dom} defines the logical structure of documents and the way a document is accessed and manipulated. In this definition, a \emph{document} is the representation of a set of information that may be stored in diverse systems which would traditionally be described as data \cite{W3C2004}. In other words, the \gls{dom} formally defines the representation layer of web application. This property is what make the \gls{dom} an ideal target to generate locators.

\subsection{Hypertext Markup Language}
\label{sec:hpath-introduction-HTML}

\gls{html} is a standard markup language derived from the \gls{xml}. It is maintained by the \gls{whatwg} and designed to display the \gls{dom} in a web page. In its current implementation, \gls{html}5, it specifies a set of more granular content models which allow to improve support for multimedia and while maintaining good machine parsing capabilities, ease its readability by humans. The specification of HTML5 can be found in the \gls{html} Living Standard documentation \cite{WHATWG2021}. 

The \gls{html} syntax consists in a tree differentiating two types of nodes: (1) encoded marker (tags) differentiating bits of information composing the document or defining anchors for multimedia content and referred to as elements $E$ and (2) any other types of nodes, $N$, containing data\footnote{text node, CDATA section node, comment node,.} or lower level structural properties\footnote{attribute node, processing instruction node, document node, document type node.} of the document.

The root of an HTML document tree is an \emph{html} element, $e_{html}$. The standard enforces the presence of two unique elements as direct children of $e_{html}$ that can appear only once in the document: a \emph{head} element, $e_{head}$ and a \emph{body} element, $e_{body}$. $e_{body}$ is the sectioning root that represents the content of the document. With \gls{html}5 a lot of semantic has been added to children elements of $e_{body}$. In this work we follow the categories presented in the HTML elements reference \cite{MDN2020,WHATWG2021} (Table~\ref{tab:hpath-introduction-html5}).

\begin{table}
\centering
\caption{Categories of elements defined by the HTML elements references. When the scope of our category differs from the HTML element references, the original merged categories are mentioned in footnote.}
\label{tab:hpath-introduction-html5}
\begin{tabular}{>{\raggedright}m{1.3in}>{\raggedright}m{3.5in}}
\toprule
\textbf{\scriptsize{Category}} & \textbf{\scriptsize{Description}}\tabularnewline
\toprule
\scriptsize{\textit{Document Metadata}} & \scriptsize{Elements encapsulating data that are not present in the page, but rather, affects the way content is presented or provides additional information about the document. \emph{e.g.} $E_{link}$, $E_{meta}$ and $E_{style}$.} \tabularnewline
\scriptsize{\textit{Content Sectioning}} & \scriptsize{Elements providing landmark to organize the content of a document into logical pieces. According to the \gls{w3c} recommendation, the structural information conveyed visually to users should be represented programmatically by the appropriate sectioning element defined by the ARIA landmark roles\cite{W3C2014}. \emph{e.g.} $E_{aside}$, $E_{footer}$, $E_{nav}$, etc.} \tabularnewline
\scriptsize{\textit{Text Content\parnote{\scriptsize{Text Content and Table Content.}}}} & \scriptsize{Elements organizing blocks of content, typically text. \emph{e.g.} $E_{li}$, $E_{p}$, $E_{blockquote}$, etc.} \tabularnewline
\scriptsize{\textit{Inline Text Semantics\parnote{\scriptsize{Inline Text Semantics and Demarcating Edits.}}}} & \scriptsize{Elements formating text such as bold, italic, etc. There are also knowon as \emph{inline} content. \emph{e.g.} $E_{em}$, $E_{strong}$, $E_{cite}$, $E_{small}$, etc.}\tabularnewline
\scriptsize{\textit{Embedded Content\parnote{\scriptsize{Embedded Content, Image \& Multimedia, SVG \& MathML and Web Components.}}}} & \scriptsize{Elements allowing to embedded multimedia resources and other content. They provide information about where to load the content from and how to display it. \emph{e.g.} $E_{img}$, $E_{object}$, $E_{embed}$, $E_{video}$, etc.} \tabularnewline
\scriptsize{\textit{Interactive\parnote{\scriptsize{Interactive and Form.}}}} & \scriptsize{Elements with which a user can interact with. Typically it represents input elements. \emph{e.g.} $E_{input}$, $E_{button}$, etc.} \tabularnewline
\scriptsize{\textit{Scripting}} & \scriptsize{Elements allowing to integrate scripting in order to create dynamic content and Web application. \emph{e.g.} $E_{canvas}$, $E_{noscript}$ and $E_{script}$} \tabularnewline
\bottomrule
\end{tabular}
\parnotes
\end{table}

A specificity of an \gls{html} document is its focus on presentation of the data. Thus, a mechanism to apply styles and positioning to elements under the sectioning root, $e_{body}$, was introduced under the form of the \gls{css}. \gls{css} defines a set of styling rules that are to be applied to target elements described by a \gls{css} selector (see Section~\ref{sec:hpath-introduction-css-selector}).

Finally, the \gls{html} standard reserves two elements with no meaning by themselves: the div element, $e_{div}$, from the Content Sectioning category and the span element, $e_{span}$, from the Inline Text Semantics category. These two types do not hold any semantic meaning. Both these elements act as placeholders to mark up user defined semantics common to their children through the use of global attributes, \emph{e.g.} class, id or name. This specific semantics can be useful when used with styling where a subtree can be targeted by a style rule or when interacting with a script to apply some processing to a portion of the document.

\begin{figure}
\centering
\caption{Example of HTML Document}
\label{fig:html-document}
\begin{minipage}{0.8\linewidth}
\begin{lstlisting}[language=HTML]
<!DOCTYPE html>
<html>
    <head>
        <style>
        .someStyle {
          border: 5px outset red;
          text-align: center;
        }
        </style>
    </head>
    <body>
        <h1>Section Header</h1>
        <div class="someStyle">
          <p id="myTarget">Text <span style="color:red">in</span> a div.</p>
        </div>
    </body>
</html>
\end{lstlisting}
\end{minipage}
\end{figure}

Figure~\ref{fig:html-document} offers an example of a \gls{html}5 Document. Line 1 defines the type of document, here notifying the parser that the document follows the \gls{html}5 standards. The element $e_{html}$ (Line 2-17) contains to children, $e_{head}$ (Line 3-9) and $e_{body}$ (Line 11-16). Within $e_{head}$ we observe the presence of an element $e_{style}$ from the \emph{Document Metadata} category, which defines some \gls{css} properties for the document. $e_{body}$ contains two elements: $e_{h1}$ (Line 12), from the category \emph{Section Content} which is used to display a title, and $e_{div}$ (Line 13-15) which does have any semantic in itself but is used to apply the style defined in the $e_{style}$ thanks to its \texttt{class} attribute (``someStyle''). Within $e_{h1}$, we observe the presence of text node, $n_{text}$ (``Section Header''). Similarly, $e_{span}$ (Line 14) applies some style to the text within it (``in'') but in this case the style is directly provided by the attribute \texttt{style}.

\subsection{DOM-based Locators}
\label{sec:hpath-introduction-locators}

A DOM-based locator can be defined as a query language for selecting a subset of targeted nodes $N_{target} \subseteq N_D$ where $N_D$ is the set of nodes contained in the document $D$. Hence, a DOM-based locator is query on a tree $D$ returning a set of nodes $N_{target} \subseteq N_D$ where the size $|N_{target}|$ varies between 0 and $|N_D|$. The \gls{html} standard describes two ways of querying a set of nodes $N_{target} \subseteq N_D$ in a \gls{html} document: XPath and \gls{css} Selector. 

\subsubsection{XPath}
\label{sec:hpath-introduction-xpath}

XML Path Language also known as XPath came from an effort to provide a common syntax and semantics to identify parts of an XML document\cite{W3C2016}. Thus, the query language is not bounded to \gls{html} document but rather to any \gls{xml} compliant document\footnote{Note that the \gls{w3c} describes a slightly modified version of the XPath 1.0 standard to interact with HTML documents.}. To query $N_{target}$, XPath relies on the concept of location path. A location path\cite{Gottlob2002} is a series of successive steps traversing the XML tree. At each step, a set of nodes, referred to as context nodes, are uniquely identified. The direct left step of the context node describe the parent node and the right step describes child nodes. During the evaluation of a context node, predicates allow to filter subset of nodes among their siblings (\emph{e.g.} all nodes with a specific attribute). Hence during the location path resolution, the path is traversed from left to right, filtering out part of the tree until it reaches the right-most step and returns the result of the query. If the left-most step is the root of the tree, we call it an absolute XPath, otherwise, if the left-most step describes the root of one or many subtrees in the document, it is said to be a relative XPath. 

Each step can be defined by a selection node criteria, an optional axis relation and an optional set of predicates\cite{Barton2003}. The selection node criteria defines the element tag from the HTML document. An axis represents a relationship to the context node, and locates relative nodes (parent, self, attributes, preceding, etc.) in the document. The predicate is an expression contained between brackets after the node criteria allowing more fine grain selection during the evaluation of the step. Predicates can express logical expressions, arithmetic operations or string manipulations \cite{Gottlob2005}.

Going back to the example presented in Figure~\ref{fig:html-document}, lets assume that $N_{target}$ is $e_{p}$. Using an absolute XPath, the query takes the following form: \texttt{/html/body/div/p}. However, note that because $e_p$ is unique in the \gls{html} document, in this case, using a relative XPath can be easy as: \texttt{//p}. Unfortunately, the introduction of a subsequent $e_p$ in the document would change the result of the latter query which would now returns more than one node. To circumvent this limitation, let's exploit the  attribute \texttt{id} of $e_p$. Thus, by relying on this attribute, we can query the document using the following XPath: \texttt{//p[@id='myTarget']}. The last XPath can be translated to: retrieve all element from the document designating a paragraph ($e_p$) for which the value of the attribute \texttt{id} equals to the text ``myTarget''.

\subsubsection{CSS Selector}
\label{sec:hpath-introduction-css-selector}

Contrarily to XPath that relies on location paths, \gls{css} Selector is a structure that determines which elements are matched in the document tree by a pattern described by the locator. Here, the locator is a chain of one or more sequences of simple selectors separated by combinators\cite{W3C2018} represented by ">". The \gls{css} standard defines 6 types of simple selectors supported by the standard: (1) type selector, (2) universal selector, (3) attribute selector, (4) class selector, (5) ID selector and (6) pseudo-class. The type selector qualify an element based on its tag, a class selector based on the class attribute of the element and an ID selector relies on the id attribute of the element. A universal selector is a wild card character (*) allowing to match any substring. The attribute selector is a generalization of the ID selector and the class selector where the name of the attribute and its values are defined. Finally, the pseudo-class concept is introduced to permit selection based on information lying outside the HTML document or that cannot be expressed using other simple selectors (\emph{e.g.} a:visited which allows to target an anchor element, $E_a$, that has already been visited).

In the example from Figure~\ref{fig:html-document}, we can observe an \gls{css} Selector at Line 5. Indeed, \texttt{.someStyle} present in $e_{style}$ is a pattern that allows to apply the style describe in Line 6-7 to all the nodes which match it, \ie\ all nodes with a \texttt{class} attribute for which the value is equal to ``someStyle''. In the case of the example, this pattern is matched by one node, $e_{div}$ at Line 13. 

Both XPath and \gls{css} Selector rely on internal properties of the elements they target by exploiting the elements attributes (id, class, etc.) which is the cause of their limited flexibility to \gls{dom} evolution. In the following section, we present HPath aiming at alleviating this limitation of DOM-based locators.