\section{Research Questions}
\label{sec:hpath-rqs}

HPath makes a strong assumption on the type of web elements it can successfully locate and properly describe. Hence, to enable the evaluation of its capability, we investigate first which web elements are predominantly targeted by GUI-based tests. Thus, we ask:

\begin{description} 
\item[\textbf{RQ1}:]
\emph{Which categories of elements are predominantly targeted by GUI-based test scripts?} 
\end{description} 

Next, we evaluate our new approach, HPath, and compare its performances in terms of properties of web elements it can leverage. We compare it against two algorithms generating XPath, namely, \emph{absolute XPath} and \emph{Robula+}. This leads to the following research question:

\begin{description} 
\item[\textbf{RQ2}:]
\emph{Which element properties are exploited by the different location path strategies?} 
\end{description} 
        
Finally, we investigate the resilience against SUT evolution of HPath and compare it to the two XPath generation algorithms used in RQ2 and express the final research question as follow:

\begin{description} 
\item[\textbf{RQ3}:]
\emph{What is the resilience against SUT evolution of the different strategies?} 
\end{description} 