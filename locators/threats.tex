\section{Threats to Validity}

The internal threat to validity concerns the way evolution is computed. During our analysis of the \gls{html} document pairs, we realized that some pairs only vary by a date or a timestamp unrelated to any of the targeted elements. While decreasing the overall percentage of breakage, because this effect is similar for all techniques, it does not have any effect on neither the ranking of the methods nor the number of breakages observed.

Furthermore, when we observe a breakage, we do not have the guarantee of the iso-functionality of the \gls{gui} element being targeted. Indeed, if the element is deeply modified or it might be expected for the locator to break. We limit the impact of this effect by using the Absolute XPath as a baseline and by ensuring that in the two versions the same test was executed against the \gls{sut}.

Finally, conducting our case study on two projects, the conclusions we draw may not hold true for other projects. This constitutes a major threat to the external validity for the generalization of our results outside the context of this study. However, the two projects have very different profiles, thus making the study covering boundary cases among the full range of possible projects. To further alleviate this limitation, we encourage other teams to replicate and extend our results.

