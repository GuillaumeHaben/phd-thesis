\section{Conclusion}
\label{sec:sherloc-conclusion}

Root-causing flaky tests is a young nonetheless important research area. Surveys showed how difficult flakiness debugging can be for developers, even when knowing which test is flaky (post-detection)~\cite{Eck2019,Habchi2022Qualitative}. 
We leveraged our results to elaborate on a few recommendations for future works.
We presented the first empirical evaluation of SBFL as a potential approach for identifying flaky classes. We investigated three approaches: pure SBFL, SBFL augmented with change and code metrics, and an ensemble of them. 
We evaluated these approaches on five open-source Java projects. Our results show that SBFL-based approaches can identify flaky classes relatively well, especially with code and change metrics, suggesting that code components responsible for flakiness exhibit similar properties with faults. This finding highlights the potential of existing fault localisation techniques for flakiness identification. At the same time, the results show that flaky tests can have unique failure causes that may mislead any coverage-based root cause analysis, stressing the need to consider these flakiness-specific causes in future studies.

Our study forms the first step towards flakiness localisation. We believe that there is a lot of room for improvement and encourage future studies to explore additional techniques, fault prediction metrics, and devise techniques that can further improve and support flakiness localisation. 
