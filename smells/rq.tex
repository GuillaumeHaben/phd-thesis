\section{Research questions}


We begin our investigation by looking at how well \gls{suit} smells are covered by both the grey literature and the academic literature. The goal behind this first question is to investigate whether or not there exists a gap between the knowledge present in the research community and among practitioners. Thus, we formulate our first research question as:

\begin{description} 
\item[\textbf{RQ1}:]
\emph{What are the \gls{suit} smells studied in academic and grey literature?} 
\end{description} 

We continue our endeavor by assessing to which extent the symptoms of each smell are present in \gls{suit}s. To that end, we compute the number of occurrences of the symptoms for each smell in every \gls{suit}. In other words, for each smell, we associate a metric and observe how that metric is distributed over the different tests. Therefore, we ask the following question:

\begin{description} 
\item[\textbf{RQ2}:]
\emph{How widespread are \gls{suit} smell symptoms in \gls{suit}s?} 
\end{description} 

Finally, we focus on the refactoring actions performed by the maintainer of the test suites to remove the smell symptoms. However, because those symptoms can be removed by actions not related to the symptom removal itself but merely being the by-product of another (\emph{e.g.} removing a feature, thus, the associated smelly code), we focus our analysis on fine-grain changes, specifically targeting the removal of the symptom itself. Hence, we formulate our third research question as follow:

\begin{description} 
\item[\textbf{RQ3}:]
\emph{How often do we observe refactoring actions removing \gls{suit} smell symptoms?} 
\end{description} 