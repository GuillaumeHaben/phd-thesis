\section{Preliminary Analysis}
\label{sec:survey-preliminary}

%The main objective of this study is to understand the measures adopted by practitioners when dealing with flaky tests.
We conduct a grey literature review (GLR) to establish an initial mapping of the measures adopted by practitioners when dealing with flaky tests.
This mapping lays the foundation for our mitigation analysis (RQ3) and helps in guiding our interview design.
With respect to this objective, this GLR is exploratory and non-exhaustive.  
In the following, we explain our process for collecting, evaluating, and analysing data from the grey literature.
\paragraph{Search}
We followed the recommendations of Kitchenham and Charters~\cite{kit_cha_2007}, for the reviewing process in general, and the guidelines of Garousi \etal~\cite{garousi2019guidelines} for the aspects specific to grey literature.
The research question for our review is:
\begin{itemize}
    \item \textsc{\textbf{RQ3:}} How do practitioners address flaky tests?
\end{itemize}
In order to answer this question, we focused our review on materials published by practitioners describing their mitigation of flakiness, \eg technical reports, presentations, blogs, etc.
To collect these materials, we queried the advanced Google search engine with the following string:
\texttt{(Mitigate OR Manage OR Deal OR Control OR Avoid OR Prevent OR Tools OR Identify OR Detect) AND (Flaky OR Intermittent OR Unreliable OR non-deterministic) AND Tests}.

This query resulted in $276,000$ results.
We manually checked the top 100 articles and only accepted articles that:
\begin{itemize}
    \item Are written by practitioners. Articles and Blog posts written by researchers are excluded.
    \item Depict practitioners' views on flakiness and do not only address the problem theoretically.
\end{itemize}
We found that only 56 articles correspond to the searched material as a large part of the top-100 articles were dedicated to the introduction of flakiness without addressing its mitigation.
% Indeed, many articles are not based on practitioner views and they only explain the issue of flaky tests theoretically.
% We also found several formal literature articles or blogs that are inspired by them.
% We excluded these articles and focused on sources that reflect practitioner practices.
%The search engine and query 
%The inclusion criteria
\paragraph{Analysis}
The objective of this step is to identify and categorise the flakiness mitigation measures from the selected articles.
For this purpose, we first examined the 56 articles to check their adequacy for our analysis.
We relied on the quality assessment checklist presented by Garousi \etal~\cite{garousi2019guidelines}, which is specifically designed for grey literature sources.
We found that three factors are particularly relevant in our context and we adopted them as exclusion criteria:
\begin{itemize}[wide=10pt,noitemsep,topsep=0pt]
    \item \textbf{Objectivity:} We exclude sources where the authors have a clear vested interest. For instance, articles that promote new tools or plugins for mitigating flaky tests are generally biased.
    \item \textbf{Method adequacy:} We found that very few sources have a clearly stated their aim and methodology. However, from the presented content, we could identify articles that were not based on practical experience and exclude them. 
    For instance, in several cases, the authors present mitigation measures from a compilation of other sources and not based on their own experiences.
    \item \textbf{Topic adequacy:} We checked whether the articles enrich our analysis or not. More specifically, we excluded articles that do not present any mitigation measures for flaky tests.
\end{itemize}
The full quality assessment is available with our artefacts~\cite{artefacts}.
Based on the three exclusion criteria, we selected 38 articles that fit within the study scope and objectives.
Two authors read these articles and iteratively synthesised a classification of the measures described by practitioners.
This consensual process is similar to the qualitative analysis performed on the interview transcripts (\textit{cf.} Section~\ref{sec:survey-interviews}).
%We found that the adopted strategies fall in three major categories: prevention of flaky tests, detection of flakiness before or after its manifestation, and mitigation actions taken when a test is identified as flaky.
The results of this analysis are presented in Table~\ref{table:strategies} and will be discussed in Section~\ref{sec:survey-results}.
Interestingly, in our grey literature analysis, we observed that the articles do not explain the rationale behind the choice of measures.
Similarly, the consequences of the measures are generally dismissed.
Hence, we try to address these gaps in our interviewing process.

