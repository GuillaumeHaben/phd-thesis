\section{Related work}
\label{sec:survey-related}

The first study on test flakiness was carried out by Luo \etal~\cite{luo_empirical_2014}. 
They analysed 201 commits from 51 open source projects in order to understand the root causes of flaky tests. % and the strategies used by developers to fix them. 
They showed that Async Waits, concurrency, and test order-dependency are the main categories of flakiness.
Later, many techniques were proposed to detect flaky tests with minimal resources, relying on reruns~\cite{lam_idflakies_2019}, coverage analysis~\cite{bell_deflaker_2018}, or static and dynamic test features~\cite{Pinto2020,alshammari2021flakeflagger,haben-msr}.
Other studies focused on highlighting the effects of flakiness on mutation testing and program repair~\cite{flakime}.

%Later Lam \etal~\cite{Lam2020} analysed 55 Java projects to understand the introduction of flakiness in software systems. 
%They found that 75\% of the 245 detected flaky tests were already flaky when added to the test suite, thus justifying the need to run detectors on newly-introduced tests. 
%The study of Luo \etal was replicated by Thorve \etal~\cite{Thorve2018} on Android applications where they found new root causes of flakiness: dependency, program Logic, and UI. 
%Dutta \etal~\cite{dutta_detecting_2020} conducted an extensive study on flaky tests (causes and fixes) in machine learning applications, finding that algorithmic non-determinism represents the biggest source of flakiness.

Several studies have been conducted to inspect flakiness in industrial contexts.
%highlighting the importance of this the challenge it represents and the necessity for solutions to this problem.
Lam \etal conducted two studies \cite{Lam2019RootCausing,lam_study_2020} about flaky tests at Microsoft. The first study showed that the number of build failures can quickly become significant despite having a low number of flaky tests. 
Thence, they introduced \textit{RootFinder}, a tool that identifies the root causes of flaky tests by analysing differences in test logs and spotting suspect method calls. 
In their second study, Lam \etal presented \textit{FaTB}, an automated tool that speeds the runtime of test suites by lowering timeouts and waits without impacting the overall test suite flake rate. 
Leong \etal~\cite{LeongSPTM19} studied flaky tests at Google and found that more than 80\% of test output transitions are caused by flakiness. 
% In their study, they also showed that flakiness systematically overestimated the performance when evaluating regression testing techniques.
At Apple, Kowalczyk \etal~\cite{Kowalczyk2020} introduced a flakiness scoring system and showed its ability to reduce flakiness by 44\%. 

Eck \etal~\cite{eck_understanding_2019} surveyed 21 Mozilla developers, asking them to classify 200 flaky tests in terms of root causes and fixing efforts. 
%They found that flakiness is perceived as a significant problem regardless of the team and project size. 
This study highlighted four new categories of flakiness: restrictive ranges, test case timeout, test suite timeout, and platform dependency.
It also provided evidence about flakiness from the CUT and showed that flaky tests can have organisational impacts.
In this study, we leverage a different qualitative approach (interviews) to address other aspects of flakiness in practice.
More specifically, we investigate broader sources of flakiness (\eg SUT and infrastructure) instead of the root causes (\eg concurrency and timeouts).
Our study also inspects the actions taken by practitioners in order to prevent, detect, and alleviate flaky tests.
Result-wise, our findings confirm the observations of Eck \etal about (i) the impact of flakiness on the test suite reliability and (ii) the challenges of reproducing and debugging flaky tests.
Furthermore, we highlight new flakiness impacts, on testing practices and product quality, and we synthesise a list of automation challenges for flakiness mitigation.


%Finally, \etal~\cite{ahmad_empirical_2019} carried out another survey of developers from 4 Swedish companies and identified 23 factors affecting flakiness.

%Add result wise
