\section{Threats to Validity}
\label{sec:survey-threats}

%\paragraph{Transferability} 
A possible threat to the generalisability of our study is the number of participants. Unfortunately, %It would have been preferable to have a larger set of participants, but
due to the specificity of the topic, it was challenging to find developers qualified to take part in the study. We tried to ensure the quality of our results by only considering practitioners with relevant experience (with flakiness in particular and testing in general). The experience of our participants ranges from 6 to 35 years, with an average of 16 years.
Our participants also constitute a diverse set of roles, company sizes, and application domains.
Moreover, %after analysing the interview transcripts, we estimated that 
the collected data are enough to answer our research questions and provide us a theoretical saturation~\cite{glaser2007remodeling}.




%\paragraph{Credibility} 
A potential threat to the credibility of our findings could be the credibility of the analysed materials as we relied on grey literature and interview transcripts.
In grey literature, we followed the quality assessment guidelines of Garousi \etal~\cite{garousi2019guidelines}, which were specifically designed for such purposes.
In interviews, we communicated the study objectives to the participants and clearly explained that the process is not judgemental.
Moreover, we formulated our questions to target the practitioner experiences and observations.% instead of the knowledge that could be acquired from external sources.




%\paragraph{Confirmability}
A potential threat to the confirmability of our results is the accuracy of the analysis of the transcripts.
To mitigate this threat, two authors performed consensual coding and all the authors discussed the coding guide iteratively, to ensure the clarity and precision of the identified sub-categories.
